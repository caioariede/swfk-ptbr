% ch2.tex
% This work is licensed under the Creative Commons Attribution-Noncommercial-Share Alike 3.0 New Zealand License.
% To view a copy of this license, visit http://creativecommons.org/licenses/by-nc-sa/3.0/nz
% or send a letter to Creative Commons, 171 Second Street, Suite 300, San Francisco, California, 94105, USA.


\chapter{8 multiplicado por 3.57 é$\ldots$}\label{ch:8multipliedby3.57}

Quanto é 8 multiplicado por 3.57? Você teria que usar uma calculadora, certo? Bom, a não ser que você seja extremamente esperto e consiga fazer mutiplicações de frações, de cabeça --- mas isso não vem ao caso. Você pode fazer isso no terminal do Python. Inicie o terminal novamente (veja o Capítulo~\ref{ch:notallsnakeswillsquishyou} para mais informações, caso você tenha pulado uma parte por algum motivo estranho), e assim que ele iniciar, digite 8$*$3.57 e pressione a tecla Enter:

\begin{listing}
\begin{verbatim}
Python 3.0 (r30:67503, Dec  6 2008, 23:22:48) 
Type "help", "copyright", "credits" or "license" for more information.
>>> 8 * 3.57
28.559999999999999
\end{verbatim}
\end{listing}

A tecla estrela (*), ou asterisco (shift 8, em alguns teclados), é usada para multiplicação\index{multiplication}, ao invés do sinal de vezes (\textsf{X}) que você usa na escola (usar o asterisco é necessário, pois caso contrário, como o computador entenderia quando você quisesse usar a letra \emph{x} ou o símbolo de multiplicação \textsf{X} ?). Que tal uma equação, que é um pouco mais interessante?

Suponhamos que você faça algumas pequenas tarefas, ao menos uma vez por semana e ganhe R\$5 para isso. E que você também entregue alguns jornais, pelo menos 5 vezes na semana e ganhe R\$30 para isso --- quanto você ganharia em um ano?

\begin{figure}[t]
\begin{center}
\fbox{\colorbox{PaleBlue}{\parbox{.75\linewidth} {
\subsection*{Python is broken!?!?}

If you just picked up a calculator and entered 8 x 3.57 the answer showing on the display will be:\\

\textsf{
28.56\\
}

\noindent
Why is Python different?  Is it broken?\\

\noindent
Actually, no. The reason can be found in the way floating point\index{floating point} (fractional numbers with a decimal place) numbers are handled by the computer.  It's a complicated and somewhat confusing problem for beginners, so it's best to just remember that when you're working with fractions (i.e. with a decimal place on a number), \emph{sometimes} the result won't be exactly what you were expecting. This is true for multiplication, division, addition or subtraction.
}}}
\end{center}
\end{figure}

Se estivessemos escrevendo isso em um papel, escreveriamos algo assim:
\begin{verbatim}
(5 + 30) x 52
\end{verbatim}

Que é R\$5 + R\$30 multiplicado por 52 semanas em um ano.  \begin{samepage}Claro, você é esperto e nós sabemos que 5 + 30 é 35, então a equação seria:

\begin{verbatim}
35 x 52
\end{verbatim}
\end{samepage}

Que é bastante fácil de se fazer com uma calculadora, ou até mesmo no papel. Mas nós podemos fazer todos esses calculos no terminal também:

\begin{listing}
\begin{verbatim}
>>> (5 + 30) * 52
1820
>>> 35 * 52
1820
\end{verbatim}
\end{listing}

Mas e se você gastasse R\$10 toda semana? Quanto lhe sobraria no final do ano? Nós poderiamos escrever essa equação no papel de diversas maneiras, mas vamos fazer isso no terminal:

\begin{listing}
\begin{verbatim}
>>> (5 + 30 - 10) * 52
1300
\end{verbatim}
\end{listing}

Isso seria R\$5 mais R\$30, menos R\$10, multiplicado pelas 52 semanas no ano. E lhe sobraria R\$1300 no final do ano. Certo, mas isso não está parecendo muito útil. Nós poderiamos fazer tudo isso com uma calculadora. Mas nós voltaremos mais tarde nesse assunto e demonstraremos como tornar isso muito mais útil.

Você pode multiplicar\index{multiplication}, somar\index{addition} (obviamente), subtrair\index{subtraction} e dividir, usando o terminal do Python, junto à diversas outras operações matemáticas que não nos aprofudaremos no momento. Para agora, os símbolos matemáticos mais básicos (chamados operadores\index{operators}) em Python, são:

\begin{center}
\begin{tabular}{|c|c|}
\hline
+ & Adição \\
\hline
- & Subtração \\
\hline
* & Multiplicação \\
\hline
/ & Divisão \\
\hline
\end{tabular}
\end{center}

O motivo da barra (/) ser utilizada para divisão, é que seria muito difícil desenhar a linha de divisão (além disso, o teclado dos computadores não contempla o símbolo de divisão $\div$) que você está acostumado a usar nas equações. Por exemplo, se você tem 100 ovos e 20 caixas e gostaria de saber quantos ovos deveriam ir em cada caixa, você dividiria 100 por 20, escrevendo a seguinte equação:

\begin{displaymath}
\frac{100}{20}
\end{displaymath}

Ou se você usa o método de Divisão Longa, você escreveria dessa forma:

\begin{displaymath}
\longdiv{100}{20}
\end{displaymath}

Ou ainda, você poderia escrever assim:

\begin{displaymath}
100 \div 20
\end{displaymath}

Porém, em Python você apenas digitaria ``100 / 20''.

\emph{Que eu acho muito mais simples. Porém eu sou apenas um livro --- do que eu sei?}

\section{Uso de parênteses e ``Prioridade de Operações''}\index{order of operations}

Nós usamos parênteses em uma linguagem de programação para controlar o que chamamos de ``Prioridade de Operações''. Uma operação é a utilização de um operador (algum daqueles símbolos da tabela acima). Existem outros operadores além dos símbolos matemáticos básicos, mas para estes operadores (adição, subtração, multiplicação e divisão), é o suficiente saber que a multiplicação e a divisão tem mais prioridade que a adição e a subtração. Que significa que você faz a multiplicação e a divisão em uma equação, antes da adição e da subtração. Na equação abaixo, onde todos os operadores são de adição (+), os números são adicionados em ordem:

\begin{listing}
\begin{verbatim}
>>> print(5 + 30 + 20)
55
\end{verbatim}
\end{listing}

\noindent
Da mesma forma, nesta equação, existem somente operadores de adição e subtração, por isso o Python considera cada número na ordem em que ele aparece.

\begin{listing}
\begin{verbatim}
>>> print(5 + 30 - 20)
15
\end{verbatim}
\end{listing}

\noindent
Mas na equação abaixo, existe um operador de multiplicação, então os números 30 e 20 são considerados primeiro. Esta equação é outro jeito de se dizer, ``multiplique 30 por 20 e então adicione 5 ao resultado'' (multiplicação primeiro, pois ela tem mais prioridade que a adição):

\begin{listing}
\begin{verbatim}
>>> print(5 + 30 * 20)
605
\end{verbatim}
\end{listing}

\noindent
Então o que acontece se nós usarmos parênteses? A seguinte equação nos mostrará o resultado:

\begin{listing}
\begin{verbatim}
>>> print((5 + 30) * 20)
700
\end{verbatim}
\end{listing}

\noindent
Por que o resultado é diferente? Porque os parênteses controlam a ordem das operações. Com os parênteses, o Python sabe que deve realizar as operações entre parênteses primeiro, e depois as que estiverem fora. Então, essa equação é outra maneira de dizer, ``adicione 5 e 30, então multiplique o resultado por 20''.

O uso dos parênteses pode se tornar complicado. Podem existir parênteses dentro de parênteses:

\begin{listing}
\begin{verbatim}
>>> print(((5 + 30) * 20) / 10)
70
\end{verbatim}
\end{listing}

\noindent
Neste caso, o Python calculará o que estiver \textbf{dentro} dos parênteses primeiro, depois o que estiver fora, e assim por diante. Então essa equação é outra maneira de se dizer, ``adicione 5 e 30, então multiplique o resultado por 20, finalmente divida o resultado por 10''. O resultado, sem parênteses, é novamente um pouco diferente:

\begin{listing}
\begin{verbatim}
>>> 5 + 30 * 20 / 10
65
\end{verbatim}
\end{listing}

Neste caso, o 30 é multiplicado por 20 primeiro, então o resultado é dividido por 10 e finalmente o 5 é adicionado ao resultado.

\emph{Lembre-se de que a multiplicação e a divisão são sempre realizadas antes da adição e da subtração, a não ser que os parênteses estejam sendo utilizados para controlar a ordem das operações.}

\section{Não há nada possa mudar tanto como uma variável}\index{variable}

Uma `variável`, em termos de programação, é usada para descrever um lugar onde se possa armazenar coisas. Essas "coisas" podem ser números, textos, listas de números e textos --- e todos os outros tipos de itens, que seria impossível de descrever aqui. Para agora, vamos apenas imaginar as variáveis como uma caixa de correios.

\begin{center}
\includegraphics*[width=76mm]{eps/girlbubble.eps}
\end{center}

Você pode colocar coisas (como uma carta ou um pacote) na caixa de correios, assim como você pode colocar coisas (números, textos, listas de números e textos, etc) em uma variável. A ideia da caixa de correios, é um modo que muitas linguagens de programação funcionam. Mas não todas.

Em Python, as variáveis são um pouco diferentes. Ao invés de uma caixa de correios com coisas dentro, uma variável lembra mais um selo que é colado do lado de fora de uma caixa de correios. Nós podemos descolar esse selo e colá-lo em outra coisa, ou até mesmo amarrá-lo em mais de uma coisa (talvez usando uma corda). Nós criamos uma variável dando à ela um nome, usando o sinal de igual (=) e então dizendo ao Python o que nós queremos apontar para aquele nome.

\begin{listing}
\begin{verbatim}
>>> fred = 100
\end{verbatim}
\end{listing}

Nós apenas criamos uma variável chamada `fred' e dissemos para que ela apontasse para o número 100. É quase o mesmo que dizer ao Python para se lembrar daquele número porque nós vamos querer usá-lo posteriormente. Para descobrir para onde uma variável está apontando, nós apenas digitamos `print' no terminal, seguido do nome da variável e então pressionamos o Enter, por exemplo:

\begin{listing}
\begin{verbatim}
>>> fred = 100
>>> print(fred)
100
\end{verbatim}
\end{listing}

Nós também podemos dizer ao Python, que nós queremos que a variável \code{fred} aponte para outra coisa:

\begin{listing}
\begin{verbatim}
>>> fred = 200
>>> print(fred)
200
\end{verbatim}
\end{listing}

\noindent
Na primeira linha, nós dizemos que agora nós queremos que o fred aponte para o número 200. Depois, na segunda linha, nós perguntamos para onde o fred está apontando, apenas para provar que mudou. Nós podemos também apontar mais de um nome para a mesma coisa:

\begin{listing}
\begin{verbatim}
>>> fred = 200
>>> john = fred
>>> print(john)
200
\end{verbatim}
\end{listing}

No código acima, nós estavamos dizendo que nós queriamos que o nome (ou etiqueta) \code{john} apontasse para a mesma coisa que \code{fred} estava apontando.
Claro que, `fred' não é um nome muito útil para uma variável. Ele não nos diz nada a respeito da sua utilidade. Uma caixa de correios é fácil --- você usa uma caixa de correios para enviar cartas. Mas uma variável pode ter diversas utilidades e pode apontar para um monte de coisas diferentes, então nós precisamos de um nome mais informativo.
\par
Suponha que você tenha iniciado o terminal do Python, digitado `fred = 200' e então tenha saído --- passou 10 anos escalando o Monte Everest, atravessando o Deserto do Saara, pulando de uma ponte na Nova Zelândia e finalmente, passeou por todo o rio Amazonas --- quando você voltou para o seu computador, como você se lembraria o que o número 200 significa (e para que ele seria usado)?

\noindent
\emph{Eu acho que eu não lembraria.}

\noindent
Eu só fiz isso até agora e não tenho ideia do que `fred = 200` significa (a não ser um \emph{nome} apontando para o número \emph{200}). Então após 10 anos, você não teria nenhuma chance de se lembrar.
\par
Aha!  But, what if we called our variable: \emph{number\_of\_students}.
Ahá! Mas e se nós chamassemos nossa variável de: \emph{numero\_de\_estudantes}.

\begin{listing}
\begin{verbatim}
>>> numero_de_estudantes = 200
\end{verbatim}
\end{listing}

Nós podemos fazer isso, pois o nome das variáveis podem conter letras, números e (\_) sublinhados --- embora elas não possam começar com um número. Se você voltar após 10 anos, `numero\_de\_estudantes' ainda vai fazer sentido. Você pode digitar:

\begin{listing}
\begin{verbatim}
>>> print(numero_de_estudantes)
200
\end{verbatim}
\end{listing}

\noindent
E você vai imediatamente se lembrar que estamos falando de 200 estudantes. Nem sempre é importante usar nomes significativos para variávies. Você pode usar qualquer coisa, desde uma única letra (como o `a') até grandes sentenças. E de vez em quando, se estiver fazendo algo rápido, um nome de variável simples e curto também será útil. Depende muito do que você vai querer pensar posteriormente, ao olhar para o nome dessa variável.

\begin{listing}
\begin{verbatim}
este_nome_de_variavel_tambem_eh_valido_porem_nao_eh_muito_util
\end{verbatim}
\end{listing}

\section{Usando uma variável}\index{Variáveis}

Agora que nós sabemos como criar uma variável, como nós a usamos? Lembra da equação que nós vimos lá atrás? Aquela para saber quanto você teria em dinheiro no final do ano, caso ganhasse R\$5 por semana fazendo alguns pequenos trabalhos, R\$30 por semana entregando jornais e gastasse R\$10 por semana. Até agora, o que temos é:

\begin{listing}
\begin{verbatim}
>>> print((5 + 30 - 10) * 52)
1300
\end{verbatim}
\end{listing}

\noindent
Que tal se transformarmos os 3 primeiros números em variáveis? Tente digitar o seguinte:

\begin{listing}
\begin{verbatim}
>>> trabalhos = 5
>>> entregas_de_jornal = 30
>>> gastos = 10
\end{verbatim}
\end{listing}

\noindent
Nós apenas criamos as variáveis `trabalhos', `entregas\_de\_jornal' e `gastos'. Nós podemos redigir a equação para ver:

\begin{listing}
\begin{verbatim}
>>> print((trabalhos + entregas_de_jornal - gastos) * 52)
1300
\end{verbatim}
\end{listing}

Que dará exatamente o mesmo resultado. E se você ganhasse mais R\$2 por semana, fazendo alguns trabalhos extras? Mude a variável `trabalhos' para 7, então aperte a tecla para cima ($\uparrow$) até que a equação apareça novamente e aperte Enter:

\begin{listing}
\begin{verbatim}
>>> trabalhos = 7
>>> print((trabalhos + entregas_de_jornal - gastos) * 52)
1404
\end{verbatim}
\end{listing}

Assim é muito mais fácil descobrir que você ficou com R\$1404 no final do ano. Você pode tentar alterar outras variáveis e apertar a seta para cima para realizar a conta novamente e você verá o efeito que terá.

\begin{listing}
\begin{verbatim}
Se você gastar o dobro do dinheiro por semana:
>>> gastos = 20
>>> print((trabalhos + entregas_de_jornal - gastos) * 52)
884
\end{verbatim}
\end{listing}

Você terá poupado apenas R\$884 no final do ano. Isso ainda é bem pouco útil. Nós ainda não chegamos em algo realmente útil. Mas para o momento, é o bastante para entender que variávies são usadas para armazenar coisas.

\noindent
\emph{Pense em uma caixa de correios com uma etiqueta colada!}

\section{Um pedaço de `string'?}\index{strings}

Se você está prestando atenção, não apenas folheando as páginas olhando as partes boas, você deve se lembrar que eu mencionei que variáveis podem ser usadas para tudo --- não somente números. Em programação, na maior parte do tempo nós chamamos um texto de `string' (corda). Que pode parecer um pouco estranho; mas se você pensar que um texto é apenas um punhado de letras `amarradas' umas as outras, talvez faça um pouco mais sentido.

\noindent
\emph{E novamente, caso não faça.}

Neste caso, tudo que você deve saber é que uma string é apenas um punhado de letras, números e outros símbolos juntos, formando algo mais significativo. Todas as letras, números e símbolos neste livro poderiam formar uma string. O seu nome pode ser uma string. Assim como o endereço da sua casa. Nosso primeiro programa em Python, no capítulo \ref{ch:notallsnakeswillsquishyou}, usou uma string: `Ola mundo'.
\par
Em Python, nós criamos uma string colocando aspas ao redor do texto. Então nós podemos pegar a nossa variável de pouca utilidade `fred' e colocar uma string dentro:

\begin{listing}
\begin{verbatim}
>>> fred = "esta é uma string"
\end{verbatim}
\end{listing}

\noindent
E nós podemos ver o que está dentro da variável \code{fred}, digitando \code{print(fred)}:

\begin{listing}
\begin{verbatim}
>>> print(fred)
esta é uma string
\end{verbatim}
\end{listing}

\noindent
Nós também podemos usar aspas simples para criar uma string:

\begin{listing}
\begin{verbatim}
>>> fred = 'esta também é uma string'
>>> print(fred)
esta também é uma string
\end{verbatim}
\end{listing}

Porém, se você tentar digitar mais de uma linha de texto na sua string, usando aspas simples (') ou duplas ("), você receberá uma mensagem de erro. Por exemplo, digite a seguinte linha e aperte Enter. Você verá uma mensagem de erro assustadora, parecida com a seguinte:

\begin{listing}
\begin{verbatim}
>>> fred = "estas são duas
  File "<stdin>", line 1
    fred = "estas são duas
                         ^
SyntaxError: EOL while scanning string literal
\end{verbatim}
\end{listing}

\index{multi-line string}Nós falaremos sobre errors mais tarde, mas por agora, se você quiser mais de uma linha de texto, você pode usar 3 aspas simples:

\begin{listing}
\begin{verbatim}
>>> fred = '''estas são duas
... linhas de texto em uma única string'''
\end{verbatim}
\end{listing}

\noindent
Imprima o conteúdo para ver se funcionou:

\begin{listing}
\begin{verbatim}
>>> print(fred)
estas são duas
linhas de texto em uma única string
\end{verbatim}
\end{listing}

By the way, you'll see those 3 dots (...) quite a few times when you're typing something that continues onto another line (like a multi line string).  In fact, you'll see it a lot more as we continue.

\section{Tricks with Strings}\label{trickswithstrings}

Here's an interesting question:  what's 10 * 5 (10 times 5)?  The answer is, of course, 50.

\noindent
\emph{All right, that's not an interesting question at all.}

But what is 10 * 'a' (10 times the letter a)?  It might seem like a nonsensical question, but here's the answer from the World of Python:

\begin{listing}
\begin{verbatim}
>>> print(10 * 'a')
aaaaaaaaaa
\end{verbatim}
\end{listing}

This works with more than just single character strings:

\begin{listing}
\begin{verbatim}
>>> print(20 * 'abcd')
abcdabcdabcdabcdabcdabcdabcdabcdabcdabcdabcdabcdabcdabcdabcdabcdabcdabcdabcdabcd
\end{verbatim}
\end{listing}

Another trick with a string, is embedding values.  You can do this by using \%s, which is like a marker (or a placeholder) for a value you want to include in a string.  It's easier to explain with an example:

\begin{listing}
\begin{verbatim}
>>> mytext = 'I am %s years old'
>>> print(mytext % 12)
I am 12 years old
\end{verbatim}
\end{listing}

In the first line, the variable mytext is created with a string containing some words and a placeholder (\%s).  The \%s is a little beacon saying ``replace me with something'' to the Python console.  So on the next line, when we call \code{print(mytext)}, we use the \% symbol, to tell Python to replace the marker with the number 12. We can reuse that string and pass in different values:

\begin{listing}
\begin{verbatim}
>>> mytext = 'Hello %s, how are you today?'
>>> name1 = 'Joe'
>>> name2 = 'Jane'
>>> print(mytext % name1)
Hello Joe, how are you today?
>>> print(mytext % name2)
Hello Jane, how are you today?
\end{verbatim}
\end{listing}

In the above example, 3 variables (mytext, name1 and name2) are created---the first includes the string with the marker.  Then we can print the variable `mytext', and again use the \% operator to pass in variables `name1' and `name2'.  You can use more than one placeholder:

\begin{listing}
\begin{verbatim}
>>> mytext = 'Hello %s and %s, how are you today?'
>>> print(mytext % (name1, name2))
Hello Joe and Jane, how are you today?
\end{verbatim}
\end{listing}

When using more than one marker, you need to wrap the replacement values with brackets---so (name1, name2) is the proper way to pass 2 variables. A set of values surrounded by brackets (the round ones, not the square ones) is called a \emph{tuple}, and is a little bit like a list, which we'll talk about next.

\section{Not quite a shopping list}\index{lists}

Eggs, milk, cheese, celery, peanut butter, and baking soda.  Which is not quite a full shopping list, but good enough for our purposes. If you wanted to store this in a variable you could create a string:

\begin{listing}
\begin{verbatim}
>>> shopping_list = 'eggs, milk, cheese, celery, peanut butter, baking soda'
>>> print(shopping_list)
eggs, milk, cheese, celery, peanut butter, baking soda
\end{verbatim}
\end{listing}

Another way would be to create a `list', which is a special kind of object in Python:

\begin{listing}
\begin{verbatim}
>>> shopping_list = [ 'eggs', 'milk', 'cheese', 'celery', 'peanut butter', 
... 'baking soda' ]
>>> print(shopping_list)
['eggs', 'milk', 'cheese', 'celery', 'peanut butter', 'baking soda']
\end{verbatim}
\end{listing}

This is more typing, but it's also more useful.  We could print the 3rd item in the list by using its position (called its index position), inside square brackets []:

\begin{listing}
\begin{verbatim}
>>> print(shopping_list[2])
cheese
\end{verbatim}
\end{listing}

Lists start at index position 0---so the first item in a list is 0, the second is 1, the third is 2.  That doesn't make a lot of sense to most people, but it does to programmers.  Pretty soon, when you walk up some stairs you'll start counting with zero rather than one.  That will really confuse your little brother or sister.
\par
We can show all the items from the 3rd item up to the 5th in the list, by using a colon inside the square brackets:

\begin{listing}
\begin{verbatim}
>>> print(shopping_list[2:5])
['cheese', 'celery', 'peanut butter']
\end{verbatim}
\end{listing}

[2:5] is the same as saying that we are interested in items from index position 2 up to (but not including) index position 5.  And, of course, because we start counting with 0, the 3rd item in the list is actually number 2, and the 5th item is actually number 4. Lists can be used to store all sorts of items.  They can store numbers:

\begin{listing}
\begin{verbatim}
>>> mylist = [ 1, 2, 5, 10, 20 ]
\end{verbatim}
\end{listing}

\noindent
And strings:

\begin{listing}
\begin{verbatim}
>>> mylist = [ 'a', 'bbb', 'ccccccc', 'ddddddddd' ]
\end{verbatim}
\end{listing}

\noindent
And mixtures of numbers and strings:

\begin{listing}
\begin{verbatim}
>>> mylist = [1, 2, 'a', 'bbb']
>>> print(mylist)
[1, 2, 'a', 'bbb']
\end{verbatim}
\end{listing}

\noindent
And even lists of lists:

\begin{listing}
\begin{verbatim}
>>> list1 = [ 'a', 'b', 'c' ]
>>> list2 = [ 1, 2, 3 ]
>>> mylist = [ list1, list2 ]
>>> print(mylist)
[['a', 'b', 'c'], [1, 2, 3]]
\end{verbatim}
\end{listing}

In the above example, a variable called `list1' is created with 3 letters, `list2' is created with a 3 numbers, and `mylist' is created using list1 and list2. Things can get rather confusing, rather quickly, if you start creating lists of lists of lists of lists$\ldots$ but luckily there's not usually much need for making things that complicated in Python. Still it is handy to know that you can store all sorts of items in a Python list.

\noindent
\emph{And not just your shopping.}

\subsection*{\color{BrickRed}Replacing items}\index{lists!replacing}

We can replace an item in the list, by setting its value in a similar way to setting the value of a normal variable. For example, we could change celery to lettuce by setting the value in index position 3:

\begin{listing}
\begin{verbatim}
>>> shopping_list[3] = 'lettuce'
>>> print(shopping_list)
['eggs', 'milk', 'cheese', 'lettuce', 'peanut butter', 'baking soda']
\end{verbatim}
\end{listing}

\subsection*{\color{BrickRed}Adding more items...}\index{lists!appending}

We can add items to a list by using a method called `append'.  A method is an action or command that tells Python that we want to do something.  We'll talk more about methods later, but for the moment, to add an item to our shopping list, we can do the following:

\begin{listing}
\begin{verbatim}
>>> shopping_list.append('chocolate bar')
>>> print(shopping_list)
['eggs', 'milk', 'cheese', 'lettuce', 'peanut butter', 'baking soda', 
'chocolate bar']
\end{verbatim}
\end{listing}

Which, if nothing else, is certainly an improved shopping list.

\subsection*{\color{BrickRed}$\ldots$and removing items}\index{lists!removing}

We can remove items from a list by using the command `del' (short for delete).  For example, to remove the 6th item in the list (baking soda):

\begin{listing}
\begin{verbatim}
>>> del shopping_list[5]
>>> print(shopping_list)
['eggs', 'milk', 'cheese', 'lettuce', 'peanut butter', 'chocolate bar']
\end{verbatim}
\end{listing}

Remember that positions start at zero, so shopping\_list[5] actually refers to the 6th item.

\subsection*{\color{BrickRed}2 lists are better than 1}\index{lists!joining}

We can join lists together by adding them, as if we were adding two numbers:

\begin{listing}
\begin{verbatim}
>>> list1 = [ 1, 2, 3 ]
>>> list2 = [ 4, 5, 6 ]
>>> print(list1 + list2)
[1, 2, 3, 4, 5, 6]
\end{verbatim}
\end{listing}

\noindent
We can also add the two lists and set the result to another variable:

\begin{listing}
\begin{verbatim}
>>> list1 = [ 1, 2, 3 ]
>>> list2 = [ 4, 5, 6 ]
>>> list3 = list1 + list2
>>> print(list3)
[1, 2, 3, 4, 5, 6]
\end{verbatim}
\end{listing}

\noindent
And you can multiply a list in the same way we multiplied a string:

\begin{listing}
\begin{verbatim}
>>> list1 = [ 1, 2 ]
>>> print(list1 * 5)
[1, 2, 1, 2, 1, 2, 1, 2, 1, 2]
\end{verbatim}
\end{listing}

\noindent
In the above example, multiplying list1 by five is another way of saying ``repeat list1 five times''. However, division (/) and subtraction (-) don't make sense when working with lists, so you'll get errors when trying the following examples:

\begin{listing}
\begin{verbatim}
>>> list1 / 20
Traceback (most recent call last):
  File "<stdin>", line 1, in <module>
TypeError: unsupported operand type(s) for /: 'list' and 'int'
\end{verbatim}
\end{listing}

\noindent
or:

\begin{listing}
\begin{verbatim}
>>> list1 - 20
Traceback (most recent call last):
  File "<stdin>", line 1, in <module>
TypeError: unsupported operand type(s) for -: 'type' and 'int'
\end{verbatim}
\end{listing}

\noindent
You'll get a rather nasty error message.

\section{Tuples and Lists}\label{tuplesandlists}\index{tuples}

A tuple (mentioned earlier) is a little bit like a list, but rather than using square brackets, you use round brackets---e.g. `(' and `)'.  You can use tuples in a similar way to a list:

\begin{listing}
\begin{verbatim}
>>> t = (1, 2, 3)
>>> print(t[1])
2
\end{verbatim}
\end{listing}

The main difference is that, unlike lists, tuples can't change, once you've created them.  So if you try to replace a value like we did earlier with the list, you'll get another error message:

\begin{listing}
\begin{verbatim}
>>> t[0] = 4
Traceback (most recent call last):
  File "<stdin>", line 1, in ?
TypeError: 'tuple' object does not support item assignment
\end{verbatim}
\end{listing}

That doesn't mean you can't change the variable containing the tuple to something else.  For example, this code will work fine:

\begin{listing}
\begin{verbatim}
>>> myvar = (1, 2, 3)
>>> myvar = [ 'a', 'list', 'of', 'strings' ]
\end{verbatim}
\end{listing}

First we create the variable \code{myvar} pointing to a tuple of 3 numbers.  Then we change \code{myvar} to point at a list of strings. This might be a bit confusing at first.  But think of it like lockers in a school.  Each locker has a name tag on it. You put something in the locker, close the door, lock it, then throw away the key.  You then peel the name tag off, wander over to another empty locker, and stick something else in that (but this time you keep the key).  A tuple is like the locked locker.  You can't change what's inside it.  But you can take the label off and stick it on an unlocked locker, and then put stuff inside that locker and take stuff out---that's the list.

\section{Things to try}

\emph{In this chapter we saw how to calculate simple mathematical equations using the Python console.  We also saw how brackets can change the result of an equation, by controlling the order that operators are used.  We found out how to tell Python to remember values for later use---using variables---plus how Python uses `strings' for storing text, and lists and tuples, for handling more than one item.}
\par

\subsection*{Exercise 1}
Make a list of your favourite toys and name it \code{toys}.  Make a list of your favourite foods and name it \code{foods}.  Join these two lists and name the result \code{favourites}.  Finally print the variable \code{favourites}.

\subsection*{Exercise 2}
If you have 3 boxes containing 25 chocolates, and 10 bags containing 32 sweets, how many sweets and chocolates do you have in total?  (Note: you can do this with one equation with the Python console)

\subsection*{Exercise 3}
Create variables for your first and last name. Now create a string and use placeholders to add your name.


\newpage
