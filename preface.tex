% preface.tex
% This work is licensed under the Creative Commons Attribution-Noncommercial-Share Alike 3.0 New Zealand License.
% To view a copy of this license, visit http://creativecommons.org/licenses/by-nc-sa/3.0/nz
% or send a letter to Creative Commons, 171 Second Street, Suite 300, San Francisco, California, 94105, USA.


\chapter*{Introdução}\normalsize
    \addcontentsline{toc}{chapter}{Preface}
\begin{center}
{\em Um recado aos Pais...}
\end{center}
\pagestyle{plain}

\noindent
Prezado(a) Pai, Mãe ou Responsável,

Para que seu filho inicie as atividades de programação, você terá que instalar o Python em seu computador. Este livro foi recentemente atualizado para utilizar o Python 3.2, que é a última versão do Python. Caso você possua uma versão antiga do Python instalada, você precisará utilizar uma versão antiga do livro.

Instalar o Python é uma tarefa muito simples, porém, dependendo do Sistema Operacional que você estiver utilizando, talvez precise passar por alguns obstáculos. Se você apenas comprou um novo computador, não tem ideia do que fazer com ele, e a última frase te deu até calafrios, você provavelmente precisará de alguém para realizar tal trabalho. Dependendo do seu computador, e da velocidade da sua conexão com a internet, essa tarefa pode levar de 10 minutos a algumas horas.

\begin{WINDOWS}

\noindent
Primeiramente, vá até \href{http://www.python.org}{www.python.org} e baixe a última versão do instalador do Python 3.2 para Windows. No momento em que estou escrevendo, é esse:
\begin{quote}
     \href{http://python.org/ftp/python/3.2/python-3.2.msi}{http://python.org/ftp/python/3.2/python-3.2.msi}
\end{quote}
Clique duas vezes sobre o ícone do instalador para Windows (você se lembra de onde o salvou, né?), e então, siga as instruções para instalá-lo no seu diretório padrão (provávelmente \emph{c:$\backslash$Python32} ou algo muito parecido).

\end{WINDOWS}

\begin{MAC}

\noindent
Primeiramente, vá até \href{http://www.python.org}{www.python.org} e baixe a última versão do instalador do Python 3.2 para Mac. No momento em que estou escrevendo, é esse:
\begin{quote}
     \href{http://python.org/ftp/python/3.2/python-3.2-macosx10.6.dmg}{http://python.org/ftp/python/3.2/python-3.2-macosx10.6.dmg}
\end{quote}
Clique duas vezes sobre o ícone do instalador para Mac (provávelmente na pasta Downloads do seu usuário), e siga as instruções para completar a instalação.

\end{MAC}

\begin{LINUX}

\noindent
Primeiramente, baixe e instale a última versão do Python 3.2 de acordo com a sua distribuição. Devido ao grande número de distribuições Linux, é impossível dar mais detalhes sobre a instalação em cada uma delas. Uma chance é que, se você está usando Linux, você deva saber o que está fazendo, e talvez até tenha se sentido insultado pela ideia de querermos dizer a você como instalar$\ldots$ tudo.

\end{LINUX}

\noindent
\emph{\color{BrickRed}Após a instalação$\ldots$}

\noindent
$\ldots$
Você pode precisar sentar ao lado de seu filho em alguns dos primeiros capítulos, mas nós esperamos que após alguns exemplos, eles tirem o teclado das suas mãos e façam eles mesmos. Eles devem pelo menos saber como utilizar algum tipo de editor de texto (não um editor como o Microsoft Word, mas um editor de texto comum, apenas para texto). Eles devem saber como abrir e fechar arquivos, criar novos arquivos de texto e salvar o que estiverem fazendo. Fora isso, o livro tentará ensinar a partir do mais básico.
\\
\noindent\\
Obrigado pela sua atenção, meus cumprimentos,
\noindent\\
O LIVRO
