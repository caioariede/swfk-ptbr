% ch4.tex
% This work is licensed under the Creative Commons Attribution-Noncommercial-Share Alike 3.0 New Zealand License.
% To view a copy of this license, visit http://creativecommons.org/licenses/by-nc-sa/3.0/nz
% or send a letter to Creative Commons, 171 Second Street, Suite 300, San Francisco, California, 94105, USA.


\chapter{Como fazer uma pergunta}\label{ch:howtoaskaquestion}

Em termos de programação, uma pergunta normalmente significa que nós queremos fazer ou uma coisa, ou outra, dependendo da resposta. Isso é chamado de \textbf{expressão if}\index{if-statement}. Por exemplo:

\begin{quotation}
Quantos anos você tem? Se você tem mais de 20 anos, você é muito velho!
\end{quotation}

Isso pode ser escrito em Python usando a seguinte expressão `if':

\begin{listing}
\begin{verbatim}
if idade > 20:
    print('você é muito velho!')
\end{verbatim}
\end{listing}

Uma expressão `if' é composta de um `if' seguido pelo que chamamos de `condição', seguido por dois-pontos (:). As linhas linhas após o `if' devem ficar em um bloco --- e se a resposta para a pergunta for `sim' (ou `True', como chamamos em programação) os comandos no bloco serão executados.
\par
A condição\index{conditions} é uma expressão de programação que retorna `sim' (True) ou `não' (False). Existem alguns símbolos (ou operadores) usados para criar condições, como:

\begin{center}
\begin{tabular}{|c|c|}
\hline
== & igual \\
\hline
!= & diferente \\
\hline
$>$ & maior que \\
\hline
$<$ & menor que \\
\hline
$>$= & maior que, ou igual \\
\hline
$<$= & menor que, ou igual \\
\hline
\end{tabular}
\end{center}

Por exemplo, se você tem 10 anos, então a condição \code{sua\_idade == 10} retornaria True (sim), mas se você não tem 10 anos, retornaria False (não). Lembre-se: não confunda os \textbf{dois} símbolos de igual, usados na condição (==), com os usados para associar valores (=) --- se você usar apenas um igual em uma \emph{condição}, você receberá uma mensagem de erro.
\par
Assumindo que você tenha a variável \code{idade} com a sua idade, então se você tiver 12 anos, a condição$\ldots$

\begin{listing}
\begin{verbatim}
idade > 10
\end{verbatim}
\end{listing}

$\ldots$ retornará `True' novamente. Se você tiver 8 anos de idade, retornará `False'. Se você tiver 10 anos de idade, retornará `False' --- pois a condição verifica se é maior que ($>$) 10, não se é maior ou igual ($>$=) à 10.

Vamos tentar alguns exemplos:

\begin{listing}
\begin{verbatim}
>>> idade = 10
>>> if idade > 10:
...     print('chegou aqui')
\end{verbatim}
\end{listing}

\noindent
Se você digitar o exemplo acima no terminal, o que deve acontecer?
\par
\noindent
Nada.
\par
\noindent
Pois o valor da variável \code{idade} não é maior que 10, o comando `print' no bloco não será executado. Que tal:

\begin{listingignore}
\begin{verbatim}
>>> idade = 10
>>> if idade >= 10:
...     print('chegou aqui')
chegou aqui
\end{verbatim}
\end{listingignore}

Se você tentar este exemplo, então você deverá ver a mensagem exibida no terminal. O mesmo ocorrerá com o próximo exemplo:

\begin{listing}
\begin{verbatim}
>>> idade = 10
>>> if idade == 10:
...     print('chegou aqui')
chegou aqui
\end{verbatim}
\end{listing}

\section{Faça isso$\ldots$ SENÃO!!!}

Nós também podemos estender a expressão `if', para que faça algo caso a condição não seja verdade. Por exemplo, exibir a palavra `Olá' no terminal, se a idade for 12, senão exibir a mensagem `Tchau'. Para fazer isso, nós precisamos usar uma expressão `se-então-senão'\index{if-then-else-statement} (o mesmo que dizer \emph{``se algo é verdadeiro, então faça \textbf{isso}, senão faça \textbf{aquilo}''}):

\begin{listing}
\begin{verbatim}
>>> idade = 12
>>> if idade == 12:
...     print('Olá')
... else:
...     print('Tchau')
Olá
\end{verbatim}
\end{listing}

Digite o exemplo acima e você verá `Olá' no terminal. Mude o valor da variável \code{idade} para outro número e `Tchau' será exibido:

\begin{listing}
\begin{verbatim}
>>> idade = 8
>>> if idade == 12:
...     print('Olá')
... else:
...     print('Tchau')

Tchau
\end{verbatim}
\end{listing}

\section{Faça isso$\ldots$ ou isso$\ldots$ ou isso$\ldots$ ou SENÃO!!!}

Nós podemos estender a expressão `if' ainda mais, usando o `elif' (abreviação de `else-if'). Por exemplo, nós podemos verificar se a idade é 10, ou se é 11, ou se é 12 e por aí em diante:

\begin{listing}
\begin{verbatim}
 1. >>> idade = 12
 2. >>> if idade == 10:
 3. ...     print('você tem 10')
 4. ... elif idade == 11:
 5. ...     print('você tem 11')
 6. ... elif idade == 12:
 7. ...     print('você tem 12')
 8. ... elif idade == 13:
 9. ...     print('você tem 13')
10. ... else:
11. ...     print('ahn?')
12. ...
13. você tem 12
\end{verbatim}
\end{listing}

No código acima, a linha 2 verifica se o valor da variável \code{idade} é igual a 10. Se não, então ele pula para a linha 4, onde é verificado se o valor da variável \code{idade} é igual a 11. Novamente, se não, então ele pula para a linha 6, para verificar se a variável é igual a 12. Neste caso é, então o Python prossegue no bloco na linha 7 e executa o comando `print'. (Felizmente você deve ter notado que existem 5 grupos neste código --- linhas 3, 5, 7, 9 e linha 11)

\section{Combinando condições}\index{conditions!combining}
Você pode combinar condições usando a palavra-chave `and' (`e', em inglês) e `or' (`ou', em inglês). Nós podemos reduzir o exemplo acima, um pouco, usando o `or' para juntar as condições:

\begin{listing}
\begin{verbatim}
1. >>> if idade == 10 or idade == 11 or idade == 12 or idade == 13:
2. ...     print('você tem %s' % idade)
3. ... else:
4. ...     print('ahn?')
\end{verbatim}
\end{listing}

Se qualquer uma das condições na linha 1 forem verdade (ex.: se idade for 10 \textbf{ou} idade for 11 \textbf{ou} idade for 12 \textbf{ou} idade for 13), então o bloco de código na linha 2 será executado, senão o Python executará a linha 4. Nós podemos reduzir um pouco o exemplo, usando o `and' e os símbolos $>$= e $<$=.

\begin{listing}
\begin{verbatim}
1. >>> if idade >= 10 and idade <= 13:
2. ...     print('você tem %s' % idade)
3. ... else:
4. ...     print('ahn?')
\end{verbatim}
\end{listing}

Felizmente você deve ter notado que se \textbf{ambas} as condições na linha 1 forem verdade, então o bloco de código na linha 2 será executado (se a idade for maior ou igual a 10 \textbf{e} a idade for menor ou igual a 13). Então  se o valor da variável idade for 12, então `você tem 12' será impresso no terminal: pois 12 é maior que 10 e também é menor que 13.

\section{Emptiness}\index{None}

There is another sort of value, that can be assigned to a variable, that we didn't talk about in the previous chapter:  \textbf{Nothing}.
\par
In the same way that numbers, strings and lists are all values that can be assigned to a variable, `nothing' is also a kind of value that can be assigned.  In Python, an empty value is referred to as \code{None} (in other programming languages, it is sometimes called null) and you can use it in the same way as other values:

\begin{listing}
\begin{verbatim}
>>> myval = None
>>> print(myval)
None
\end{verbatim}
\end{listing}

None is a way to reset a variable back to being un-used, or can be a way to create a variable without setting its value before it is used.
\par
For example, if your football team were raising funds for new uniforms, and you were adding up how much money had been raised, you might want to wait until all the team had returned with the money before you started adding it all up.  In programming terms, we might have a variable for each member of the team, and then set all the variables to None:

\begin{listing}
\begin{verbatim}
>>> player1 = None
>>> player2 = None
>>> player3 = None
\end{verbatim}
\end{listing}

We could then use an if-statement, to check these variables, to determine if all the members of the team had returned with the money they'd raised:

\begin{listing}
\begin{verbatim}
>>> if player1 is None or player2 is None or player3 is None:
...     print('Please wait until all players have returned')
... else:
...     print('You have raised %s' % (player1 + player2 + player3))
\end{verbatim}
\end{listing}

The if-statement checks whether any of the variables have a value of \code{None}, and prints the first message if they do.  If each variable has a real value, then the second message is printed with the total money raised. If you try this code out with all variables set to None, you'll see the first message (don't forget to create the variables first or you'll get an error message):

\begin{listing}
\begin{verbatim}
>>> if player1 is None or player2 is None or player3 is None:
...     print('Please wait until all players have returned')
... else:
...     print('You have raised %s' % (player1 + player2 + player3))
Please wait until all players have returned
\end{verbatim}
\end{listing}

Even if we set one or two of the variables, we'll still get the message:

\begin{listing}
\begin{verbatim}
>>> player1 = 100
>>> player3 = 300
>>> if player1 is None or player2 is None or player3 is None:
...     print('Please wait until all players have returned')
... else:
...     print('You have raised %s' % (player1 + player2 + player3))
Please wait until all players have returned
\end{verbatim}
\end{listing}

\noindent
Finally, once all variables are set, you'll see the message in the second block:

\begin{listing}
\begin{verbatim}
>>> player1 = 100
>>> player3 = 300
>>> player2 = 500
>>> if player1 is None or player2 is None or player3 is None:
...     print('Please wait until all players have returned')
... else:
...     print('You have raised %s' % (player1 + player2 + player3))
You have raised 900
\end{verbatim}
\end{listing}

\section{What's the difference$\ldots$?}\label{whatsthedifference}\index{equality}

What's the difference between \code{10} and \code{'10'}?
\par
Not much apart from the quotes, you might be thinking.  Well, from reading the earlier chapters, you know that the first is a number and the second is a string. This makes them differ more than you might expect.  Earlier we compared the value of a variable (age) to a number in an if-statement:

\begin{listing}
\begin{verbatim}
>>> if age == 10:
...     print('you are 10')
\end{verbatim}
\end{listing}

If you set variable age to 10, the print statement will be called:

\begin{listing}
\begin{verbatim}
>>> age = 10
>>> if age == 10:
...     print('you are 10')
...
you are 10
\end{verbatim}
\end{listing}

However, if age is set to \code{'10'} (note the quotes), then it won't:

\begin{listing}
\begin{verbatim}
>>> age = '10'
>>> if age == 10:
...     print('you are 10')
...
\end{verbatim}
\end{listing}

Why is the code in the block not run?  Because a string is different from a number, even if they look the same:

\begin{listing}
\begin{verbatim}
>>> age1 = 10
>>> age2 = '10'
>>> print(age1)
10
>>> print(age2)
10
\end{verbatim}
\end{listing}

See!  They look exactly the same.  Yet, because one is a string, and the other is a number, they are different values. Therefore age == 10 (age equals 10) will never be true, if the value of the variable is a string.
\par
Probably the best way to think about it, is to consider 10 books and 10 bricks.  The number of items might be the same, but you couldn't say that 10 books are exactly the same as 10 bricks, could you? Luckily in Python we have magic functions which can turn strings into numbers and numbers into strings (even if they won't quite turn bricks into books). For example, to convert the string '10' into a number you would use the function \code{int}:

\begin{listing}
\begin{verbatim}
>>> age = '10'
>>> converted_age = int(age)
\end{verbatim}
\end{listing}

\noindent
The variable converted\_age now holds the number 10, and not a string. To convert a number into a string, you would use the function \code{str}:

\begin{listing}
\begin{verbatim}
>>> age = 10
>>> converted_age = str(age)
\end{verbatim}
\end{listing}

\noindent
converted\_age now holds the string 10, and not a number. Back to that if-statement which prints nothing:

\begin{listing}
\begin{verbatim}
>>> age = '10'
>>> if age == 10:
...     print('you are 10')
...
\end{verbatim}
\end{listing}

\noindent
If we convert the variable \emph{before} we check, then we'll get a different result:

\begin{listing}
\begin{verbatim}
>>> age = '10'
>>> converted_age = int(age)
>>> if converted_age == 10:
...     print('you are 10')
...
you are 10
\end{verbatim}
\end{listing}

\newpage
