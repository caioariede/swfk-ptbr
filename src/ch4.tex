% ch4.tex
% This work is licensed under the Creative Commons Attribution-Noncommercial-Share Alike 3.0 New Zealand License.
% To view a copy of this license, visit http://creativecommons.org/licenses/by-nc-sa/3.0/nz
% or send a letter to Creative Commons, 171 Second Street, Suite 300, San Francisco, California, 94105, USA.


\chapter{Como fazer uma pergunta}\label{ch:howtoaskaquestion}

Em termos de programação, uma pergunta normalmente significa que nós queremos fazer ou uma coisa, ou outra, dependendo da resposta. Isso é chamado de \textbf{expressão if}\index{Expressão `if'}. Por exemplo:

\begin{quotation}
Quantos anos você tem? Se você tem mais de 20 anos, você é muito velho!
\end{quotation}

Isso pode ser escrito em Python usando a seguinte expressão `if':

\begin{listing}
\begin{verbatim}
if idade > 20:
    print('você é muito velho!')
\end{verbatim}
\end{listing}

Uma expressão `if' é composta de um `if' seguido pelo que chamamos de `condição', seguido por dois-pontos (:). As linhas linhas após o `if' devem ficar em um bloco --- e se a resposta para a pergunta for `sim' (ou `True', como chamamos em programação) os comandos no bloco serão executados.
\par
A condição\index{Condições} é uma expressão de programação que retorna `sim' (True) ou `não' (False). Existem alguns símbolos (ou operadores) usados para criar condições, como:

\begin{center}
\begin{tabular}{|c|c|}
\hline
== & igual \\
\hline
!= & diferente \\
\hline
$>$ & maior que \\
\hline
$<$ & menor que \\
\hline
$>$= & maior que, ou igual \\
\hline
$<$= & menor que, ou igual \\
\hline
\end{tabular}
\end{center}

Por exemplo, se você tem 10 anos, então a condição \code{sua\_idade == 10} retornaria True (sim), mas se você não tem 10 anos, retornaria False (não). Lembre-se: não confunda os \textbf{dois} símbolos de igual, usados na condição (==), com os usados para associar valores (=) --- se você usar apenas um igual em uma \emph{condição}, você receberá uma mensagem de erro.
\par
Assumindo que você tenha a variável \code{idade} com a sua idade, então se você tiver 12 anos, a condição$\ldots$

\begin{listing}
\begin{verbatim}
idade > 10
\end{verbatim}
\end{listing}

$\ldots$ retornará `True' novamente. Se você tiver 8 anos de idade, retornará `False'. Se você tiver 10 anos de idade, retornará `False' --- pois a condição verifica se é maior que ($>$) 10, não se é maior ou igual ($>$=) à 10.

Vamos tentar alguns exemplos:

\begin{listing}
\begin{verbatim}
>>> idade = 10
>>> if idade > 10:
...     print('chegou aqui')
\end{verbatim}
\end{listing}

\noindent
Se você digitar o exemplo acima no terminal, o que deve acontecer?
\par
\noindent
Nada.
\par
\noindent
Pois o valor da variável \code{idade} não é maior que 10, o comando `print' no bloco não será executado. Que tal:

\begin{listingignore}
\begin{verbatim}
>>> idade = 10
>>> if idade >= 10:
...     print('chegou aqui')
chegou aqui
\end{verbatim}
\end{listingignore}

Se você tentar este exemplo, então você deverá ver a mensagem exibida no terminal. O mesmo ocorrerá com o próximo exemplo:

\begin{listing}
\begin{verbatim}
>>> idade = 10
>>> if idade == 10:
...     print('chegou aqui')
chegou aqui
\end{verbatim}
\end{listing}

\section{Faça isso$\ldots$ SENÃO!!!}

Nós também podemos estender a expressão `if', para que faça algo caso a condição não seja verdade. Por exemplo, exibir a palavra `Olá' no terminal, se a idade for 12, senão exibir a mensagem `Tchau'. Para fazer isso, nós precisamos usar uma expressão `se-então-senão'\index{Expressão if-then-else} (o mesmo que dizer \emph{``se algo é verdadeiro, então faça \textbf{isso}, senão faça \textbf{aquilo}''}):

\begin{listing}
\begin{verbatim}
>>> idade = 12
>>> if idade == 12:
...     print('Olá')
... else:
...     print('Tchau')
Olá
\end{verbatim}
\end{listing}

Digite o exemplo acima e você verá `Olá' no terminal. Mude o valor da variável \code{idade} para outro número e `Tchau' será exibido:

\begin{listing}
\begin{verbatim}
>>> idade = 8
>>> if idade == 12:
...     print('Olá')
... else:
...     print('Tchau')

Tchau
\end{verbatim}
\end{listing}

\section{Faça isso$\ldots$ ou isso$\ldots$ ou isso$\ldots$ ou SENÃO!!!}

Nós podemos estender a expressão `if' ainda mais, usando o `elif' (abreviação de `else-if'). Por exemplo, nós podemos verificar se a idade é 10, ou se é 11, ou se é 12 e por aí em diante:

\begin{listing}
\begin{verbatim}
 1. >>> idade = 12
 2. >>> if idade == 10:
 3. ...     print('você tem 10')
 4. ... elif idade == 11:
 5. ...     print('você tem 11')
 6. ... elif idade == 12:
 7. ...     print('você tem 12')
 8. ... elif idade == 13:
 9. ...     print('você tem 13')
10. ... else:
11. ...     print('ahn?')
12. ...
13. você tem 12
\end{verbatim}
\end{listing}

No código acima, a linha 2 verifica se o valor da variável \code{idade} é igual a 10. Se não, então ele pula para a linha 4, onde é verificado se o valor da variável \code{idade} é igual a 11. Novamente, se não, então ele pula para a linha 6, para verificar se a variável é igual a 12. Neste caso é, então o Python prossegue no bloco na linha 7 e executa o comando `print'. (Felizmente você deve ter notado que existem 5 grupos neste código --- linhas 3, 5, 7, 9 e linha 11)

\section{Combinando condições}\index{Condições!Combinação}
Você pode combinar condições usando a palavra-chave `and' (`e', em inglês) e `or' (`ou', em inglês). Nós podemos reduzir o exemplo acima, um pouco, usando o `or' para juntar as condições:

\begin{listing}
\begin{verbatim}
1. >>> if idade == 10 or idade == 11 or idade == 12 or idade == 13:
2. ...     print('você tem %s' % idade)
3. ... else:
4. ...     print('ahn?')
\end{verbatim}
\end{listing}

Se qualquer uma das condições na linha 1 forem verdade (ex.: se idade for 10 \textbf{ou} idade for 11 \textbf{ou} idade for 12 \textbf{ou} idade for 13), então o bloco de código na linha 2 será executado, senão o Python executará a linha 4. Nós podemos reduzir um pouco o exemplo, usando o `and' e os símbolos $>$= e $<$=.

\begin{listing}
\begin{verbatim}
1. >>> if idade >= 10 and idade <= 13:
2. ...     print('você tem %s' % idade)
3. ... else:
4. ...     print('ahn?')
\end{verbatim}
\end{listing}

Felizmente você deve ter notado que se \textbf{ambas} as condições na linha 1 forem verdade, então o bloco de código na linha 2 será executado (se a idade for maior ou igual a 10 \textbf{e} a idade for menor ou igual a 13). Então  se o valor da variável idade for 12, então `você tem 12' será impresso no terminal: pois 12 é maior que 10 e também é menor que 13.

\section{Vácuo}\index{O valor None}

Existe outro tipo de valor, que pode ser atribuído à uma variável, que não falamos sobre nos capítulos anteriores: \textbf{Nada}.
\par
Da mesma forma que números, strings e listas podem ser atribuídos à uma variável, `nada' também é um valor que pode ser atribuído. Em Python, um valor vazio é referenciado como \code{None} (em outras linguagens de programação, muitas vezes chamado de `null') e você pode usar da mesma maneira que os outros valores:

\begin{listing}
\begin{verbatim}
>>> minhavar = None
>>> print(minhavar)
None
\end{verbatim}
\end{listing}

`None' é uma forma de apagar o conteúdo de uma variável, ou apenas de criar uma variável definindo um valor padrão antes de ser usada.
\par
Por exemplo, se o seu time de futebol estiver levantando fundos para um novo uniforme e você está somando quanto já foi levantado, você pode querer esperar até que todo time dê o dinheiro antes de iniciar a contagem. Em termos de programação, nós poderiamos ter uma variável para cada membro do time e então atribuir todas as variáveis para `None':

\begin{listing}
\begin{verbatim}
>>> jogador1 = None
>>> jogador2 = None
>>> jogador3 = None
\end{verbatim}
\end{listing}

Nós podemos usar a expressão `if', para verificar essas variáveis e determinar se todos os membros do time já retornaram com o dinheiro que levantaram:

\begin{listing}
\begin{verbatim}
>>> if jogador1 is None or jogador2 is None or jogador3 is None:
...     print('Por favor, aguarde até que todos jogadores tenham retornado')
... else:
...     print('Você conseguiu %s' % (jogador1 + jogador2 + jogador3))
\end{verbatim}
\end{listing}

A expressão `if' verifica se alguma das variáveis possui o valor \code{None}, e imprime a primeira mensagem, se sim. Se cada uma das variáveis possuir um valor, então a segunda mensagem é impressa, com o valor total levantado. Se você tentar este código com todas as variáveis com `None', você verá a primeira mensagem (não se esqueça de criar as variáveis primeiro, senão você receberá uma mensagem de erro):

\begin{listing}
\begin{verbatim}
>>> if jogador1 is None or jogador2 is None or jogador3 is None:
...     print('Por favor, aguarde até que todos jogadores tenham retornado')
... else:
...     print('Você conseguiu %s' % (jogador1 + jogador2 + jogador3))
Por favor, aguarde até que todos jogadores tenham retornado
\end{verbatim}
\end{listing}

Até mesmo se nós definirmos uma ou duas variáveis, nós ainda receberemos a mensagem:

\begin{listing}
\begin{verbatim}
>>> jogador1 = 100
>>> jogador3 = 300
>>> if jogador1 is None or jogador2 is None or jogador3 is None:
...     print('Por favor, aguarde até que todos jogadores tenham retornado')
... else:
...     print('Você conseguiu %s' % (jogador1 + jogador2 + jogador3))
Por favor, aguarde até que todos jogadores tenham retornado
\end{verbatim}
\end{listing}

\noindent
Finalmente, assim que todas variáveis estiverem definidas, você verá a mensagem do segundo bloco:

\begin{listing}
\begin{verbatim}
>>> jogador1 = 100
>>> jogador3 = 300
>>> jogador2 = 500
>>> if jogador1 is None or jogador2 is None or jogador3 is None:
...     print('Por favor, aguarde até que todos jogadores tenham retornado')
... else:
...     print('Você conseguiu %s' % (jogador1 + jogador2 + jogador3))
Você conseguiu 900
\end{verbatim}
\end{listing}

\section{Qual é a diferença$\ldots$?}\label{whatsthedifference}\index{Igualdade}

Qual é a diferença entre \code{10} e \code{'10'}?
\par
Nada além das aspas, você deve estar pensando. Bem, pelo que foi lido nos primeiros capítulos, você sabe que o primeiro é um número e o segundo é uma `string'. Isso os diferencia mais do que você deve estar pensando. Mais cedo, nós comparamos o valor da variável (idade) à um número em uma expressão `if':

\begin{listing}
\begin{verbatim}
>>> if idade == 10:
...     print('você tem 10')
\end{verbatim}
\end{listing}

Se você definir a variável idade com 10, o comando `print' será chamado:

\begin{listing}
\begin{verbatim}
>>> idade = 10
>>> if idade == 10:
...     print('você tem 10')
...
você tem 10
\end{verbatim}
\end{listing}

Porém, se `idade' for definida como \code{'10'} (note as aspas), não será:

\begin{listing}
\begin{verbatim}
>>> idade = '10'
>>> if idade == 10:
...     print('você tem 10')
...
\end{verbatim}
\end{listing}

Por que o código no bloco não foi executado? Porque uma string é diferente de um número, mesmo que eles se pareçam:

\begin{listing}
\begin{verbatim}
>>> idade1 = 10
>>> idade2 = '10'
>>> print(idade1)
10
>>> print(idade2)
10
\end{verbatim}
\end{listing}

Veja! Eles parecem exatamente iguais. Ainda, por um ser uma string e outro um número, eles tem valores diferentes. Portanto \code{idade == 10} (idade igual à 10) nunca será verdade, se o valor da variável é uma string.
\par
Provavelmente, a melhor forma de pensar sobre isso é considerar 10 livros e 10 tijolos. O número de itens pode ser o mesmo, mas você não poderia dizer que 10 livros são 10 tijolos, poderia? Felizmente em Python, nós temos funções mágicas que podem transformar strings em números e números em strings (mesmo que elas não transformem tijolos em livros). Por exemplo, para converter uma string '10' em um número, você pode usar a função \code{int}:

\begin{listing}
\begin{verbatim}
>>> idade = '10'
>>> idade_convertida = int(idade)
\end{verbatim}
\end{listing}

\noindent
A variável \code{idade\_convertida} agora contém o número 10 e não uma string. Para converter um número para uma string, você pode usar a função \code{str}:

\begin{listing}
\begin{verbatim}
>>> idade = 10
>>> idade_convertida = str(idade)
\end{verbatim}
\end{listing}

\noindent
A variável \code{idade\_convertida} agora contém a string 10 e não um número. De volta àquela expressão `if' que não imprime nada:

\begin{listing}
\begin{verbatim}
>>> age = '10'
>>> if age == 10:
...     print('you are 10')
...
\end{verbatim}
\end{listing}

\noindent
Se nós convertermos a variável \emph{antes} de verificarmos, então veremos a diferença no resultado:

\begin{listing}
\begin{verbatim}
>>> idade = '10'
>>> idade_convertida = int(idade)
>>> if idade_convertida == 10:
...     print('você tem 10')
...
você tem 10
\end{verbatim}
\end{listing}

\newpage
