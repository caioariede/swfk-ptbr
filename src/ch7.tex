% ch7.tex
% This work is licensed under the Creative Commons Attribution-Noncommercial-Share Alike 3.0 New Zealand License.
% To view a copy of this license, visit http://creativecommons.org/licenses/by-nc-sa/3.0/nz
% or send a letter to Creative Commons, 171 Second Street, Suite 300, San Francisco, California, 94105, USA.


\chapter{Um curto capítulo sobre Arquivos}\label{ch:ashortchapteraboutfiles}\index{Funções!file}

Você provavelmente já sabe o que é um arquivo.
\par
\noindent
Se os seus pais tem um escritório em casa, existe uma grande chance de eles terem algum tipo de armário de arquivo. Vários papéis importantes (principalmente coisas chatas de adultos) são armazenadas nesses armários, normalmente em pastas rotuladas com as letras do alfabeto ou meses do ano. Os arquivos em um computador são bastante semelhantes à essas pastas. Eles têm rótulos (o nome do arquivo) e são usados para armazenar informações importantes. As gavetas de um armário de arquivo, são similares aos diretórios (ou pastas) em um computador.
\par
Nós já criamos um objeto de arquivo usando Python, no capítulo anterior. O exemplo se parecia com esse:

\begin{WINDOWS}

\begin{listing}
\begin{verbatim}
>>> f = open('c:\\teste.txt')
>>> print(f.read())
\end{verbatim}
\end{listing}

\end{WINDOWS}

\begin{MAC}

\begin{listing}
\begin{verbatim}
>>> f = open('Desktop/teste.txt')
>>> print(f.read())
\end{verbatim}
\end{listing}

\end{MAC}

\begin{LINUX}

\begin{listing}
\begin{verbatim}
>>> f = open('Desktop/teste.txt')
>>> print(f.read())
\end{verbatim}
\end{listing}
 
\end{LINUX}

Um objeto de arquivo não tem somente a função \code{read}\index{Funções!file!read}. Afinal, armários de arquivo não seriam tão úteis se você só pudesse abrir as gavetas e tirar os papéis de dentro, sem poder guarda-los de volta. Nós podemos criar um novo arquivo em branco, passando um outro parâmetro quando chamamos a função \code{file}:

\begin{listing}
\begin{verbatim}
>>> f = open('novoarquivo.txt', 'w')
\end{verbatim}
\end{listing}

O 'w' é uma forma de dizer ao Python que nós queremos apenas escrever no objeto de arquivo e não ler. Nós agora podemos guardar informações no arquivo, usando a função \code{write}\index{Funções!file!write}.

\begin{listing}
\begin{verbatim}
>>> f = open('novoarquivo.txt', 'w')
>>> f.write('este é um arquivo de teste')
\end{verbatim}
\end{listing}

E então, precisamos dizer ao Python que nós terminamos e não queremos mais escrever no arquivo --- nós usamos a função \code{close}\index{Funções!file!close} para isso.

\begin{listing}
\begin{verbatim}
>>> f = open('novoarquivo.txt', 'w')
>>> f.write('este é um arquivo de teste')
>>> f.close()
\end{verbatim}
\end{listing}

Se você abrir o arquivo usando o seu editor de texto favorito, verá que ele contém o texto: ``este é um arquivo de teste''. Ou melhor ainda, nós podemos usar o Python para ler:

\begin{listing}
\begin{verbatim}
>>> f = open('novoarquivo.txt')
>>> print(f.read())
este é um arquivo de teste
\end{verbatim}
\end{listing}

\newpage
