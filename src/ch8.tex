% ch8.tex
% This work is licensed under the Creative Commons Attribution-Noncommercial-Share Alike 3.0 New Zealand License.
% To view a copy of this license, visit http://creativecommons.org/licenses/by-nc-sa/3.0/nz
% or send a letter to Creative Commons, 171 Second Street, Suite 300, San Francisco, California, 94105, USA.


\chapter{Mais e mais tartarugas}\index{turtle!advanced}\label{ch:turtlesgalore}

Voltando ao módulo \code{turtle}, que começamos a ver no Capítulo~\ref{ch:turtles}. Lembre-se que para configurar a tela para a tartaruga desenhar, nós precisamos importar o módulo e criar um objeto `Pen':

\begin{listing}
\begin{verbatim}
>>> import turtle
>>> t = turtle.Pen()
\end{verbatim}
\end{listing}

Agora nós podemos usar algumas funções básicas para mover a tartaruga pela tela e desenhar algumas formas simples, mas seria interessante usar algo que nós já abordamos anteriormente. Por exemplo, o código que nós usamos para criar o quadrado, foi:

\begin{listing}
\begin{verbatim}
>>> t.forward(50)
>>> t.left(90)
>>> t.forward(50)
>>> t.left(90)
>>> t.forward(50)
>>> t.left(90)
\end{verbatim}
\end{listing} 

\noindent
Nós podemos reescrever esse código, usando um laço `for':

\begin{listing}
\begin{verbatim}
>>> t.reset()
>>> for x in range(1, 5):
...     t.forward(50)
...     t.left(90)
...
\end{verbatim}
\end{listing}

Que de fato é muito menos para se digitar, mas para algo ainda mais interessante, tente o seguinte:

\begin{listing}
\begin{verbatim}
>>> t.reset()
>>> for x in range(1, 9):
...     t.forward(100)
...     t.left(225)
...
\end{verbatim}
\end{listing}

Este código produz uma estrela de 8 pontas, como exibido na figura~\ref{fig20} (a tartaruga vira 225 graus, cada vez que anda 100 pixels para frente).

\begin{figure}
\begin{center}
\includegraphics[width=72mm]{eps/figure20.eps}
\end{center}
\caption{A tartaruga desenhando uma estrela de 8 pontas.}\label{fig20}
\end{figure}

\noindent
Em um ângulo diferente (175 graus), e um laço mais longo (37 vezes), nós podemos fazer uma estrela com ainda mais pontas (como visto na figura~\ref{fig21}):

\begin{listing}
\begin{verbatim}
>>> t.reset()
>>> for x in range(1, 38):
...     t.forward(100)
...     t.left(175)
...
\end{verbatim}
\end{listing}

\begin{figure}
\begin{center}
\includegraphics[width=72mm]{eps/figure21.eps}
\end{center}
\caption{Uma estrela com muito mais pontas.}\label{fig21}
\end{figure}

\noindent
Ou, que tal o seguinte código, que produz uma estrela tipo espiral, na figura~\ref{fig22}.

\begin{listing}
\begin{verbatim}
>>> for x in range(1, 20):
...     t.forward(100)
...     t.left(95)
...
\end{verbatim}
\end{listing}

\begin{figure}
\begin{center}
\includegraphics[width=72mm]{eps/figure22.eps}
\end{center}
\caption{Uma estrela tipo espiral.}\label{fig22}
\end{figure}

\noindent
Aqui está algo um pouco mais complicado:

\begin{listing}
\begin{verbatim}
>>> t.reset()
>>> for x in range(1, 19):
...     t.forward(100)
...     if x % 2 == 0:
...         t.left(175)
...     else:
...         t.left(225)
...
\end{verbatim}
\end{listing}

No código acima, nós verificamos se a variável \code{x} contém um número par. Para fazer isso, nós usamos um operador chamado módulo (\%)\index{modulo operator}, na expressão: \code{x \% 2 == 0}.
\par
x \% 2 é igual a zero, se o número na variável \code{x} puder ser dividido por dois, sem sobrar nada --- se isso não faz muito sentido, não se preocupe, apenas lembre de usar `\code{x \% 2 == 0}' para verificar se o número contido em uma variável é um número par. O resultado desse código, é uma estrela de 9 pontas, como na figura~\ref{fig23}.

\begin{figure}
\begin{center}
\includegraphics[width=84mm]{eps/figure23.eps}
\end{center}
\caption{Uma estrela de 9 pontas.}\label{fig23}
\end{figure}

Você não precisa apenas desenhar estrelas e simples formas geométricas. Usando uma combinação de chamadas de função, a sua tartaruga pode desenhar muitas coisas diferentes. Por exemplo:

\begin{listing}
\begin{verbatim}
t.color(1, 0, 0)
t.begin_fill()
t.forward(100)
t.left(90)
t.forward(20)
t.left(90)
t.forward(20)
t.right(90)
t.forward(20)
t.left(90)
t.forward(60)
t.left(90)
t.forward(20)
t.right(90)
t.forward(20)
t.left(90)
t.forward(20)
t.end_fill()
t.color(0, 0, 0)
t.up()
t.forward(10)
t.down()
t.begin_fill()
t.circle(10)
t.end_fill()
t.setheading(0)
t.up()
t.forward(90)
t.right(90)
t.forward(10)
t.setheading(0)
t.begin_fill()
t.down()
t.circle(10)
t.end_fill()
\end{verbatim}
\end{listing}

\noindent
Que é uma forma beeeem longa de se desenhar o carrinho feio e com visual bem primitivo, da figura~\ref{fig24}. Mas demonstra várias outras funções da tartaruga: \code{color}, para alterar a cor da caneta usada pela tartaruga, \code{fill}, que preenche uma área da tela; e \code{circle}, para desenhar um círculo com um determinado tamanho.

\begin{figure}
\begin{center}
\includegraphics[width=80mm]{eps/figure24.eps}
\end{center}
\caption{A tartaruga é péssima em desenhar carros!}\label{fig24}
\end{figure}

\section{Colorindo}

A função \code{color}\index{turtle!color}, aceita 3 parâmetros. O primeiro parâmetro é o valor para o vermelho, o segundo é o valor para o verde e o último é o valor para o azul.
\par
\emph{Por que vermelho, verde e azul?}
\par
Se você já brincou com diferentes cores de tinta, você provavelmente já sabe parte da resposta para essa questão. Quando você mistura duas cores diferentes, você consegue outra cor\footnote{Na realidade, as três cores \textbf{primárias} são o vermelho, o amarelo e o azul e não vermelho/verde/azul (RGB -- Red/Green/Blue), no computador.}. Quando você mistura azul e vermelho, você consegue roxo$\ldots$ e quando você mistura várias cores, normalmente você consegue um marrom cor de lama. Em um computador, você pode misturar várias cores, da mesma forma --- coloca vermelho e verde juntos para conseguir amarelo --- exceto que com um computador, nós estamos combinando cores de luz, não cores de tinta.
 
Mesmo que não estejamos usando tinta, por um momento, pense em 3 potes grandes de tinta. Um vermelho, um verde e um azul. Os potes estão cheios, então nós vamos dizer que um pote cheio tem o valor de 1 (ou 100\%). Nós então derramamos todo o pote vermelho (100\%) em um tanque, seguido de toda a tinta verde (novamente, 100\%). Depois de misturar um pouco, nós teremos a cor amarela. Vamos desenhar um círculo amarelo usando a tartaruga:

\begin{listing}
\begin{verbatim}
>>> t.color(1, 1, 0)
>>> t.begin_fill()
>>> t.circle(50)
>>> t.end_fill()
\end{verbatim}
\end{listing}

Então acima, nós chamamos a função `color' com 100\% de vermelho (red), 100\% de verde (green) e 0\% de azul (blue) -- em outras palavras, 1, 1 e 0. Para facilitar o teste de outras cores, vamos transformar isso em uma função:

\begin{listing}
\begin{verbatim}
>>> def meucirculo(red, green, blue):
...     t.color(red, green, blue)
...     t.begin_fill()
...     t.circle(50)
...     t.end_fill()
...
\end{verbatim}
\end{listing}

\noindent
Nós podemos desenhar um círculo verde brilhante, usando toda a tinta verde (1 ou 100\%):

\begin{listing}
\begin{verbatim}
>>> mycircle(0, 1, 0)
\end{verbatim}
\end{listing}

\noindent
E nós podemos desenhar um círculo verde escuro, usando metade de toda a tinta verde (0.5 ou 50\%):

\begin{listing}
\begin{verbatim}
>>> meucirculo(0, 0.5, 0)
\end{verbatim}
\end{listing}

Aqui é onde pensar sobre tinta não faz muito sentido mais. No mundo real, se você tiver um pote de tinta verde, não importa o quanto você use, você sempre terá o mesmo tom. Com cores em um computador, devido a estarmos brincando com luzes, usando menos de uma cor geralmente resultará em uma tonalidade mais escura. É o mesmo que você acender uma tocha durante a noite, você terá uma luz amarelada --- quando a chama e a luz começarem a diminuir, a cor amarela começará a ficar mais e mais escura. Apenas para você ver, tente desenhar um círculo com um vermelho completo e um vermelho pela metade (1 e 0.5) e um azul completo e um azul pela metade.

\begin{listing}
\begin{verbatim}
>>> meucirculo(1, 0, 0)
>>> meucirculo(0.5, 0, 0)

>>> meucirculo(0, 0, 1)
>>> meucirculo(0, 0, 0.5)
\end{verbatim}
\end{listing}

\noindent
Diferentes combinações de vermelho, verde e azul produzirão uma enorme variedade de cores. Você terá a cor de ouro, usando 100\% do vermelho, 85\% do verde e nada do azul:

\begin{listing}
\begin{verbatim}
>>> meucirculo(1, 0.85, 0)
\end{verbatim}
\end{listing}

\noindent
Uma cor rosa claro pode ser feita combinando 100\% do vermelho, 70\% do verde e 75\% do azul:

\begin{listing}
\begin{verbatim}
>>> meucirculo(1, 0.70, 0.75)
\end{verbatim}
\end{listing}

\noindent
E você consegue a cor laranja, combinando 100\% do vermelho, 65\% do verde; e o marrom com 60\% do vermelho, 30\% do verde e 15\% do azul:

\begin{listing}
\begin{verbatim}
>>> meucirculo(1, 0.65, 0)
>>> meucirculo(0.6, 0.3, 0.15)
\end{verbatim}
\end{listing}

\noindent
Não se esqueça, você pode limpar a tela usando o \code{t.clear()}.

\section{Escuridão}\index{turtle!color!black}

Aqui vai uma pergunta para você: O que acontece quando você desliga todas as cores a noite? Fica tudo preto.
\par
A mesma coisa acontece com as cores em um computador. Nenhuma luz é a mesma coisa que nenhuma cor. Então um círculo com 0 de vermelho, 0 de verde e 0 de azul:

\begin{listing}
\begin{verbatim}
>>> meucirculo(0, 0, 0)
\end{verbatim}
\end{listing}

Produz o círculo preto da figura~\ref{fig25}.

\begin{figure}
\begin{center}
\includegraphics[width=85mm]{eps/figure25.eps}
\end{center}
\caption{Um buraco negro!}\label{fig25}
\end{figure}

O recíproca é verdadeira; se você usar 100\% do vermelho, 100\% do verde e 100\% do azul, você terá branco. Use o seguinte código e o círculo preto sumirá novamente:

\begin{listing}
\begin{verbatim}
>>> meucirculo(1, 1, 1)
\end{verbatim}
\end{listing}

\section{Preenchendo coisas}\index{turtle!fill}

Você provavelmente já descobriu que a função `fill' é ligada passando o parâmetro `1', então desligada passando o parâmetro `0'. Quando você a desliga, a função na realidade preenche a área que você desenhou --- assumindo que você tenha desenhado pelo menos parte de uma forma. Então nós podemos facilmente desenhar um quadrado preenchido, usando o código que nós criamos previamente. Primeiro, vamos transformá-lo em uma função. Para desenhar um quadrado com a tartaruga, nós fazemos:

\begin{listing}
\begin{verbatim}
>>> t.forward(50)
>>> t.left(90)
>>> t.forward(50)
>>> t.left(90)
>>> t.forward(50)
>>> t.left(90)
>>> t.forward(50)
>>> t.left(90)
\end{verbatim}
\end{listing}

Então, em uma função nós podemos querer passar o tamanho do quadrado por parâmetro. Isso torna a função um pouco mais flexível:

\begin{listing}
\begin{verbatim}
>>> def meuquadrado(tamanho):
...     t.forward(tamanho)
...     t.left(90)
...     t.forward(tamanho)
...     t.left(90)
...     t.forward(tamanho)
...     t.left(90)
...     t.forward(tamanho)
...     t.left(90)
\end{verbatim}
\end{listing}

\noindent
Nós podemos testar a nossa função, chamando:

\begin{listing}
\begin{verbatim}
>>> meuquadrado(50)
\end{verbatim}
\end{listing}

Já é um começo, mas não é algo perfeito. Se você olhar para o código acima, você verá um padrão. Nós repetimos: \code{forward(tamanho)} e \code{left(90)} 4 vezes. Que é um disperdício de digitação. Nós podemos usar um laço `for' para fazer isso por nós (muito parecido com o que já fizemos anteriormente):

\begin{listing}
\begin{verbatim}
>>> def meuquadrado(tamanho):
...     for x in range(0, 4):
...         t.forward(tamanho)
...         t.left(90)
\end{verbatim}
\end{listing}

Isso é uma versão bem melhor que a anterior. Você pode testar a função com diferentes tamanhos:

\begin{listing}
\begin{verbatim}
>>> t.reset()
>>> meuquadrado(25)
>>> meuquadrado(50)
>>> meuquadrado(75)
>>> meuquadrado(100)
>>> meuquadrado(125)
\end{verbatim}
\end{listing}

E a tartaruga deverá desenhar algo como a figura~\ref{fig26}.

\begin{figure}
\begin{center}
\includegraphics[width=72mm]{eps/figure26.eps}
\end{center}
\caption{Vários quadrados.}\label{fig26}
\end{figure}

\noindent
Agora, nós podemos tentar um quadrado preenchido. Primeiro de tudo, vamos apagar a tela novamente:

\begin{listing}
\begin{verbatim}
>>> t.reset()
\end{verbatim}
\end{listing}

\noindent
Então, ativando o preenchimento e chamando a função meuquadrado novamente:

\begin{listing}
\begin{verbatim}
>>> t.begin_fill()
>>> meuquadrado(50)
\end{verbatim}
\end{listing}

\noindent
Você continuará vendo um quadrado vazio, até desativar o preenchimento:

\begin{listing}
\begin{verbatim}
>>> t.end_fill()
\end{verbatim}
\end{listing}

\noindent
Que produz algo parecido com o quadrado da figura~\ref{fig27}.

\begin{figure}
\begin{center}
\includegraphics[width=72mm]{eps/figure27.eps}
\end{center}
\caption{Um quadrado preto.}\label{fig27}
\end{figure}

Que tal se mudarmos a função, para que pudessemos desenhar tanto um quadrado preenchido como um sem preenchimento? Nós precisamos de um outro parâmetro, e um código um pouquinho mais complicado para fazer isso:

\begin{listing}
\begin{verbatim}
>>> def meuquadrado(tamanho, preenchido):
...    if preenchido == True:
...        t.begin_fill()
...    for x in range(0, 4):
...        t.forward(tamanho)
...        t.left(90)
...    if preenchido == True:
...        t.end_fill()
...
\end{verbatim}
\end{listing}

As primeiras duas linhas, verificam se o valor do parâmetro `preenchido' é igual a `True'. Se for, então ativa o preenchimento. Então nós fazemos um laço 4 vezes, para desenhar os quatro lados do quadrado e então, após isso, verificamos novamente se o parâmetro `preenchido' é igual a `True', se sim, desativamos o preenchimento. Agora você pode desenhar um quadrado preenchido, chamando:

\begin{listing}
\begin{verbatim}
>>> mysquare(50, True)
\end{verbatim}
\end{listing}

\noindent
E um quadrado não preenchido, chamando:

\begin{listing}
\begin{verbatim}
>>> meuquadrado(150, False)
\end{verbatim}
\end{listing}

\noindent
Que faz com que a nossa tartaruga desenhe a imagem da figura~\ref{fig28}$\ldots$ $\ldots$que agora que eu vi, parece um olho quadrado estranho.

\begin{figure}
\begin{center}
\includegraphics[width=72mm]{eps/figure28.eps}
\end{center}
\caption{Um olho quadrado.}\label{fig28}
\end{figure}

Você pode desenhar diversas formas e preenche-las com cores. Vamos tornar a estrela que nós criamos anteriormente, em uma função. O código original era mais ou menos assim:

\begin{listing}
\begin{verbatim}
>>> for x in range(1, 19):
...     t.forward(100)
...     if x % 2 == 0:
...         t.left(175)
...     else:
...         t.left(225)
...
\end{verbatim}
\end{listing}

Nós podemos usar as mesmas expressões `if' da função `meuquadrado' e passar os parâmetros `tamanho' e `preenchido':

\begin{listing}
\begin{verbatim}
1.  >>> def minhaestrela(tamanho, preenchido):
2.  ...     if preenchido:
3.  ...         t.begin_fill()
4.  ...     for x in range(1, 19):
5.  ...         t.forward(tamanho)
6.  ...         if x % 2 == 0:
7.  ...             t.left(175)
8.  ...         else:
9.  ...             t.left(225)
10. ...     if preenchido:
11. ...         t.end_fill()
\end{verbatim}
\end{listing}

Nas linhas 2 e 3, nós ativamos o preenchimento, dependendo do valor do parâmetro \code{preenchido} (ativado, caso seja `True' ou desativado, caso seja `False'). E nas linhas 10 e 11 nós desativamos, caso o preenchimento esteja ativado. A outra diferença nessa função, é que nós passamos o tamanho da estrela através do parâmetro `tamanho' e usamos esse valor na linha 5.
\par
Agora, vamos definir a cor para ouro (lembre-se que, para conseguirmos a cor ouro, precisamos de 100\% de vermelho, 85\% de verde e nada de azul) e então chamar a função:

\begin{listing}
\begin{verbatim}
>>> t.color(1, 0.85, 0)
>>> minhaestrela(120, True)
\end{verbatim}
\end{listing}

\noindent
A tartaruga deve desenhar a estrela cor de ouro da figura~\ref{fig29}. Nós podemos contornar a estrela trocando as cores novamente (dessa vez para preto) e redesenhando a estrela com o preenchimento desativado:

\begin{figure}
\begin{center}
\includegraphics[width=85mm]{eps/figure29.eps}
\end{center}
\caption{Uma estrela cor de ouro.}\label{fig29}
\end{figure}

\begin{listing}
\begin{verbatim}
>>> t.color(0,0,0)
>>> minhaestrela(120, False)
\end{verbatim}
\end{listing}

\noindent
Portanto, a estrela agora parece com a da figura~\ref{fig30}.

\begin{figure}
\begin{center}
\includegraphics[width=85mm]{eps/figure30.eps}
\end{center}
\caption{A star with an outline.}\label{fig30}
\end{figure}

\section{Things to try}

\emph{In this chapter we learned about the turtle module, using it to draw a few basic geometric shapes. We used functions in order to re-use some of our code, to make it easier to draw shapes with different colours.}

\subsection*{Exercise 1}
We've drawn stars, squares and rectangles.  How about an octagon?  An octagon is an 8 sided shape.
(Hint: try turning 45 degrees).

\subsection*{Exercise 2}
Now convert the octagon drawing code into a function which will fill it with a colour.

\newpage
