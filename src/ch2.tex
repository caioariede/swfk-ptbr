% ch2.tex
% This work is licensed under the Creative Commons Attribution-Noncommercial-Share Alike 3.0 New Zealand License.
% To view a copy of this license, visit http://creativecommons.org/licenses/by-nc-sa/3.0/nz
% or send a letter to Creative Commons, 171 Second Street, Suite 300, San Francisco, California, 94105, USA.


\chapter{8 multiplicado por 3.57 é$\ldots$}\label{ch:8multipliedby3.57}

Quanto é 8 multiplicado por 3.57? Você teria que usar uma calculadora, certo? Bom, a não ser que você seja extremamente esperto e consiga fazer mutiplicações de frações, de cabeça --- mas isso não vem ao caso. Você pode fazer isso no terminal do Python. Inicie o terminal novamente (veja o Capítulo~\ref{ch:notallsnakeswillsquishyou} para mais informações, caso você tenha pulado uma parte por algum motivo estranho), e assim que ele iniciar, digite 8$*$3.57 e pressione a tecla Enter:

\begin{listing}
\begin{verbatim}
Python 3.0 (r30:67503, Dec  6 2008, 23:22:48) 
Type "help", "copyright", "credits" or "license" for more information.
>>> 8 * 3.57
28.559999999999999
\end{verbatim}
\end{listing}

A tecla estrela (*), ou asterisco (shift 8, em alguns teclados), é usada para multiplicação\index{Multiplicação}, ao invés do sinal de vezes (\textsf{X}) que você usa na escola (usar o asterisco é necessário, pois caso contrário, como o computador entenderia quando você quisesse usar a letra \emph{x} ou o símbolo de multiplicação \textsf{X} ?). Que tal uma equação, que é um pouco mais interessante?

Suponhamos que você faça algumas pequenas tarefas, ao menos uma vez por semana e ganhe R\$5 para isso. E que você também entregue alguns jornais, pelo menos 5 vezes na semana e ganhe R\$30 para isso --- quanto você ganharia em um ano?

\begin{figure}[t]
\begin{center}
\fbox{\colorbox{PaleBlue}{\parbox{.75\linewidth} {
\subsection*{O Python é defeituoso!?!?}

Se você pegar uma calculadora e digitar 8 x 3.57, a resposta exibida será:\\

\textsf{
28.56\\
}

\noindent
Mas porque no Python é diferente? É um defeito?\\

\noindent
Na verdade não. O motivo é a forma com que os pontos flutuantes\index{Ponto flutuante} (frações de números com casas decimais) números são manuseados pelo computador. É um problema complicado e um pouco confuso para iniciantes, então o melhor coisa a se fazer é sempre que estiver trabalhando com frações (ex.: com números com casas decimais), se lembrar que \emph{algumas} vezes o resultado pode ser diferente do esperado. Isso é fato para multiplicação, divisão, adição ou subtração.
}}}
\end{center}
\end{figure}

Se estivessemos escrevendo isso em um papel, escreveriamos algo assim:
\begin{verbatim}
(5 + 30) x 52
\end{verbatim}

Que é R\$5 + R\$30 multiplicado por 52 semanas em um ano.  \begin{samepage}Claro, você é esperto e nós sabemos que 5 + 30 é 35, então a equação seria:

\begin{verbatim}
35 x 52
\end{verbatim}
\end{samepage}

Que é bastante fácil de se fazer com uma calculadora, ou até mesmo no papel. Mas nós podemos fazer todos esses calculos no terminal também:

\begin{listing}
\begin{verbatim}
>>> (5 + 30) * 52
1820
>>> 35 * 52
1820
\end{verbatim}
\end{listing}

Mas e se você gastasse R\$10 toda semana? Quanto lhe sobraria no final do ano? Nós poderiamos escrever essa equação no papel de diversas maneiras, mas vamos fazer isso no terminal:

\begin{listing}
\begin{verbatim}
>>> (5 + 30 - 10) * 52
1300
\end{verbatim}
\end{listing}

Isso seria R\$5 mais R\$30, menos R\$10, multiplicado pelas 52 semanas no ano. E lhe sobraria R\$1300 no final do ano. Certo, mas isso não está parecendo muito útil. Nós poderiamos fazer tudo isso com uma calculadora. Mas nós voltaremos mais tarde nesse assunto e demonstraremos como tornar isso muito mais útil.

Você pode multiplicar\index{Multiplicação}, somar\index{Adição} (obviamente), subtrair\index{Subtração} e dividir, usando o terminal do Python, junto à diversas outras operações matemáticas que não nos aprofudaremos no momento. Para agora, os símbolos matemáticos mais básicos (chamados operadores\index{Operadores}) em Python, são:

\begin{center}
\begin{tabular}{|c|c|}
\hline
+ & Adição \\
\hline
- & Subtração \\
\hline
* & Multiplicação \\
\hline
/ & Divisão \\
\hline
\end{tabular}
\end{center}

O motivo da barra (/) ser utilizada para divisão, é que seria muito difícil desenhar a linha de divisão (além disso, o teclado dos computadores não contempla o símbolo de divisão $\div$) que você está acostumado a usar nas equações. Por exemplo, se você tem 100 ovos e 20 caixas e gostaria de saber quantos ovos deveriam ir em cada caixa, você dividiria 100 por 20, escrevendo a seguinte equação:

\begin{displaymath}
\frac{100}{20}
\end{displaymath}

Ou se você usa o método de Divisão Longa, você escreveria dessa forma:

\begin{displaymath}
\longdiv{100}{20}
\end{displaymath}

Ou ainda, você poderia escrever assim:

\begin{displaymath}
100 \div 20
\end{displaymath}

Porém, em Python você apenas digitaria ``100 / 20''.

\emph{Que eu acho muito mais simples. Porém eu sou apenas um livro --- do que eu sei?}

\section{Uso de parênteses e ``Prioridade de Operações''}\index{Prioridade de Operações}

Nós usamos parênteses em uma linguagem de programação para controlar o que chamamos de ``Prioridade de Operações''. Uma operação é a utilização de um operador (algum daqueles símbolos da tabela acima). Existem outros operadores além dos símbolos matemáticos básicos, mas para estes operadores (adição, subtração, multiplicação e divisão), é o suficiente saber que a multiplicação e a divisão tem mais prioridade que a adição e a subtração. Que significa que você faz a multiplicação e a divisão em uma equação, antes da adição e da subtração. Na equação abaixo, onde todos os operadores são de adição (+), os números são adicionados em ordem:

\begin{listing}
\begin{verbatim}
>>> print(5 + 30 + 20)
55
\end{verbatim}
\end{listing}

\noindent
Da mesma forma, nesta equação, existem somente operadores de adição e subtração, por isso o Python considera cada número na ordem em que ele aparece.

\begin{listing}
\begin{verbatim}
>>> print(5 + 30 - 20)
15
\end{verbatim}
\end{listing}

\noindent
Mas na equação abaixo, existe um operador de multiplicação, então os números 30 e 20 são considerados primeiro. Esta equação é outro jeito de se dizer, ``multiplique 30 por 20 e então adicione 5 ao resultado'' (multiplicação primeiro, pois ela tem mais prioridade que a adição):

\begin{listing}
\begin{verbatim}
>>> print(5 + 30 * 20)
605
\end{verbatim}
\end{listing}

\noindent
Então o que acontece se nós usarmos parênteses? A seguinte equação nos mostrará o resultado:

\begin{listing}
\begin{verbatim}
>>> print((5 + 30) * 20)
700
\end{verbatim}
\end{listing}

\noindent
Por que o resultado é diferente? Porque os parênteses controlam a ordem das operações. Com os parênteses, o Python sabe que deve realizar as operações entre parênteses primeiro, e depois as que estiverem fora. Então, essa equação é outra maneira de dizer, ``adicione 5 e 30, então multiplique o resultado por 20''.

O uso dos parênteses pode se tornar complicado. Podem existir parênteses dentro de parênteses:

\begin{listing}
\begin{verbatim}
>>> print(((5 + 30) * 20) / 10)
70
\end{verbatim}
\end{listing}

\noindent
Neste caso, o Python calculará o que estiver \textbf{dentro} dos parênteses primeiro, depois o que estiver fora, e assim por diante. Então essa equação é outra maneira de se dizer, ``adicione 5 e 30, então multiplique o resultado por 20, finalmente divida o resultado por 10''. O resultado, sem parênteses, é novamente um pouco diferente:

\begin{listing}
\begin{verbatim}
>>> 5 + 30 * 20 / 10
65
\end{verbatim}
\end{listing}

Neste caso, o 30 é multiplicado por 20 primeiro, então o resultado é dividido por 10 e finalmente o 5 é adicionado ao resultado.

\emph{Lembre-se de que a multiplicação e a divisão são sempre realizadas antes da adição e da subtração, a não ser que os parênteses estejam sendo utilizados para controlar a ordem das operações.}

\section{Não há nada possa mudar tanto como uma variável}\index{Variável}

Uma `variável`, em termos de programação, é usada para descrever um lugar onde se possa armazenar coisas. Essas "coisas" podem ser números, textos, listas de números e textos --- e todos os outros tipos de itens, que seria impossível de descrever aqui. Para agora, vamos apenas imaginar as variáveis como uma caixa de correios.

\begin{center}
\includegraphics*[width=76mm]{eps/girlbubble.eps}
\end{center}

Você pode colocar coisas (como uma carta ou um pacote) na caixa de correios, assim como você pode colocar coisas (números, textos, listas de números e textos, etc) em uma variável. A ideia da caixa de correios, é um modo que muitas linguagens de programação funcionam. Mas não todas.

Em Python, as variáveis são um pouco diferentes. Ao invés de uma caixa de correios com coisas dentro, uma variável lembra mais um selo que é colado do lado de fora de uma caixa de correios. Nós podemos descolar esse selo e colá-lo em outra coisa, ou até mesmo amarrá-lo em mais de uma coisa (talvez usando uma corda). Nós criamos uma variável dando à ela um nome, usando o sinal de igual (=) e então dizendo ao Python o que nós queremos apontar para aquele nome.

\begin{listing}
\begin{verbatim}
>>> fred = 100
\end{verbatim}
\end{listing}

Nós apenas criamos uma variável chamada `fred' e dissemos para que ela apontasse para o número 100. É quase o mesmo que dizer ao Python para se lembrar daquele número porque nós vamos querer usá-lo posteriormente. Para descobrir para onde uma variável está apontando, nós apenas digitamos `print' no terminal, seguido do nome da variável e então pressionamos o Enter, por exemplo:

\begin{listing}
\begin{verbatim}
>>> fred = 100
>>> print(fred)
100
\end{verbatim}
\end{listing}

Nós também podemos dizer ao Python, que nós queremos que a variável \code{fred} aponte para outra coisa:

\begin{listing}
\begin{verbatim}
>>> fred = 200
>>> print(fred)
200
\end{verbatim}
\end{listing}

\noindent
Na primeira linha, nós dizemos que agora nós queremos que o fred aponte para o número 200. Depois, na segunda linha, nós perguntamos para onde o fred está apontando, apenas para provar que mudou. Nós podemos também apontar mais de um nome para a mesma coisa:

\begin{listing}
\begin{verbatim}
>>> fred = 200
>>> john = fred
>>> print(john)
200
\end{verbatim}
\end{listing}

No código acima, nós estavamos dizendo que nós queriamos que o nome (ou etiqueta) \code{john} apontasse para a mesma coisa que \code{fred} estava apontando.
Claro que, `fred' não é um nome muito útil para uma variável. Ele não nos diz nada a respeito da sua utilidade. Uma caixa de correios é fácil --- você usa uma caixa de correios para enviar cartas. Mas uma variável pode ter diversas utilidades e pode apontar para um monte de coisas diferentes, então nós precisamos de um nome mais informativo.
\par
Suponha que você tenha iniciado o terminal do Python, digitado `fred = 200' e então tenha saído --- passou 10 anos escalando o Monte Everest, atravessando o Deserto do Saara, pulando de uma ponte na Nova Zelândia e finalmente, passeou por todo o rio Amazonas --- quando você voltou para o seu computador, como você se lembraria o que o número 200 significa (e para que ele seria usado)?

\noindent
\emph{Eu acho que eu não lembraria.}

\noindent
Eu só fiz isso até agora e não tenho ideia do que `fred = 200` significa (a não ser um \emph{nome} apontando para o número \emph{200}). Então após 10 anos, você não teria nenhuma chance de se lembrar.
\par
Ahá! Mas e se nós chamassemos nossa variável de: \emph{numero\_de\_estudantes}.

\begin{listing}
\begin{verbatim}
>>> numero_de_estudantes = 200
\end{verbatim}
\end{listing}

Nós podemos fazer isso, pois o nome das variáveis podem conter letras, números e (\_) sublinhados --- embora elas não possam começar com um número. Se você voltar após 10 anos, `numero\_de\_estudantes' ainda vai fazer sentido. Você pode digitar:

\begin{listing}
\begin{verbatim}
>>> print(numero_de_estudantes)
200
\end{verbatim}
\end{listing}

\noindent
E você vai imediatamente se lembrar que estamos falando de 200 estudantes. Nem sempre é importante usar nomes significativos para variávies. Você pode usar qualquer coisa, desde uma única letra (como o `a') até grandes sentenças. E de vez em quando, se estiver fazendo algo rápido, um nome de variável simples e curto também será útil. Depende muito do que você vai querer pensar posteriormente, ao olhar para o nome dessa variável.

\begin{listing}
\begin{verbatim}
este_nome_de_variavel_tambem_eh_valido_porem_nao_eh_muito_util
\end{verbatim}
\end{listing}

\section{Usando uma variável}\index{Variável}

Agora que nós sabemos como criar uma variável, como nós a usamos? Lembra da equação que nós vimos lá atrás? Aquela para saber quanto você teria em dinheiro no final do ano, caso ganhasse R\$5 por semana fazendo alguns pequenos trabalhos, R\$30 por semana entregando jornais e gastasse R\$10 por semana. Até agora, o que temos é:

\begin{listing}
\begin{verbatim}
>>> print((5 + 30 - 10) * 52)
1300
\end{verbatim}
\end{listing}

\noindent
Que tal se transformarmos os 3 primeiros números em variáveis? Tente digitar o seguinte:

\begin{listing}
\begin{verbatim}
>>> trabalhos = 5
>>> entregas_de_jornal = 30
>>> gastos = 10
\end{verbatim}
\end{listing}

\noindent
Nós apenas criamos as variáveis `trabalhos', `entregas\_de\_jornal' e `gastos'. Nós podemos redigir a equação para ver:

\begin{listing}
\begin{verbatim}
>>> print((trabalhos + entregas_de_jornal - gastos) * 52)
1300
\end{verbatim}
\end{listing}

Que dará exatamente o mesmo resultado. E se você ganhasse mais R\$2 por semana, fazendo alguns trabalhos extras? Mude a variável `trabalhos' para 7, então aperte a tecla para cima ($\uparrow$) até que a equação apareça novamente e aperte Enter:

\begin{listing}
\begin{verbatim}
>>> trabalhos = 7
>>> print((trabalhos + entregas_de_jornal - gastos) * 52)
1404
\end{verbatim}
\end{listing}

Assim é muito mais fácil descobrir que você ficou com R\$1404 no final do ano. Você pode tentar alterar outras variáveis e apertar a seta para cima para realizar a conta novamente e você verá o efeito que terá.

\begin{listing}
\begin{verbatim}
Se você gastar o dobro do dinheiro por semana:
>>> gastos = 20
>>> print((trabalhos + entregas_de_jornal - gastos) * 52)
884
\end{verbatim}
\end{listing}

Você terá poupado apenas R\$884 no final do ano. Isso ainda é bem pouco útil. Nós ainda não chegamos em algo realmente útil. Mas para o momento, é o bastante para entender que variávies são usadas para armazenar coisas.

\noindent
\emph{Pense em uma caixa de correios com uma etiqueta colada!}

\section{Um pedaço de `string'?}\index{Strings}

Se você está prestando atenção, não apenas folheando as páginas olhando as partes boas, você deve se lembrar que eu mencionei que variáveis podem ser usadas para tudo --- não somente números. Em programação, na maior parte do tempo nós chamamos um texto de `string' (corda). Que pode parecer um pouco estranho; mas se você pensar que um texto é apenas um punhado de letras `amarradas' umas as outras, talvez faça um pouco mais sentido.

\noindent
\emph{E novamente, caso não faça.}

Neste caso, tudo que você deve saber é que uma string é apenas um punhado de letras, números e outros símbolos juntos, formando algo mais significativo. Todas as letras, números e símbolos neste livro poderiam formar uma string. O seu nome pode ser uma string. Assim como o endereço da sua casa. Nosso primeiro programa em Python, no capítulo \ref{ch:notallsnakeswillsquishyou}, usou uma string: `Ola mundo'.
\par
Em Python, nós criamos uma string colocando aspas ao redor do texto. Então nós podemos pegar a nossa variável de pouca utilidade `fred' e colocar uma string dentro:

\begin{listing}
\begin{verbatim}
>>> fred = "esta é uma string"
\end{verbatim}
\end{listing}

\noindent
E nós podemos ver o que está dentro da variável \code{fred}, digitando \code{print(fred)}:

\begin{listing}
\begin{verbatim}
>>> print(fred)
esta é uma string
\end{verbatim}
\end{listing}

\noindent
Nós também podemos usar aspas simples para criar uma string:

\begin{listing}
\begin{verbatim}
>>> fred = 'esta também é uma string'
>>> print(fred)
esta também é uma string
\end{verbatim}
\end{listing}

Porém, se você tentar digitar mais de uma linha de texto na sua string, usando aspas simples (') ou duplas ("), você receberá uma mensagem de erro. Por exemplo, digite a seguinte linha e aperte Enter. Você verá uma mensagem de erro assustadora, parecida com a seguinte:

\begin{listing}
\begin{verbatim}
>>> fred = "estas são duas
  File "<stdin>", line 1
    fred = "estas são duas
                         ^
SyntaxError: EOL while scanning string literal
\end{verbatim}
\end{listing}

\index{Strings de multilinha}Nós falaremos sobre errors mais tarde, mas por agora, se você quiser mais de uma linha de texto, você pode usar 3 aspas simples:

\begin{listing}
\begin{verbatim}
>>> fred = '''estas são duas
... linhas de texto em uma única string'''
\end{verbatim}
\end{listing}

\noindent
Imprima o conteúdo para ver se funcionou:

\begin{listing}
\begin{verbatim}
>>> print(fred)
estas são duas
linhas de texto em uma única string
\end{verbatim}
\end{listing}

A propósito, você verá esses 3 pontinhos (...) sempre que estiver digitando algo que continue na próxima linha (como as strings de várias linhas). Na verdade, você ainda verá muitos destes conforme continuamos.

\section{Truques com Strings}\label{trickswithstrings}

Aqui está uma questão interessante: quanto é 10 * 5 (10 vezes 5)? A resposta é 50, claro.

\noindent
\emph{Ok, essa não é uma questão das mais interessantes.}

Mas quanto é 10 * 'a' (10 vezes a letra a)? Isso pode parecer uma questão sem sentido, mas aqui está a resposta diretamente do Mundo do Python:

\begin{listing}
\begin{verbatim}
>>> print(10 * 'a')
aaaaaaaaaa
\end{verbatim}
\end{listing}

Isso funciona ainda com mais caracteres:

\begin{listing}
\begin{verbatim}
>>> print(20 * 'abcd')
abcdabcdabcdabcdabcdabcdabcdabcdabcdabcdabcdabcdabcdabcdabcdabcdabcdabcdabcdabcd
\end{verbatim}
\end{listing}

Outro truque com uma string é embutir valores. Você pode fazer isso usando o sinalizador \%s, que será substituido pelo valor que você quer incluir na string. É mais fácil de explicar com um exemplo:

\begin{listing}
\begin{verbatim}
>>> meutexto = 'Eu tenho %s anos de idade'
>>> print(meutexto % 12)
Eu tenho 12 anos de idade
\end{verbatim}
\end{listing}

Na primeira linha, a variável `meutexto' foi criada contendo uma string com algumas palavras e o sinalizador (\%s). O \%s está dizendo ``substitua-me com algo'' para o terminal do Python. Então na linha seguinte, quando nós chamamos \code{print(meutexto)}, nós usamos o símbolo \%, que diz ao Python para substituir o sinalizador pelo número 12. Nós podemos reutilizar a string e passar diferentes valores:

\begin{listing}
\begin{verbatim}
>>> meutexto = 'Olá %s, como vai você hoje?'
>>> nome1 = 'Joe'
>>> nome2 = 'Jane'
>>> print(meutexto % name1)
Olá Joe, como vai você hoje?
>>> print(meutexto % name2)
Olá Jane, como vai você hoje?
\end{verbatim}
\end{listing}

No exemplo acima, 3 variáveis (meutexto, nome1 e nome2) foram criadas --- o primeiro inclui a string com o sinalizador. Em seguida nós imprimimos a variável `meutexto' e usamos novamente o operador \% para passar as variáveis `nome1' e `nome2'. E você pode usar mais de um sinalizador:

\begin{listing}
\begin{verbatim}
>>> meutexto = 'Olá %s e %s, como vocês estão hoje?'
>>> print(meutexto % (nome1, nome2))
Olá Joe a Jane, como vocês estão hoje?
\end{verbatim}
\end{listing}

Quando utilizar mais de um sinalizador, você precisa envolver os valores entre parênteses --- então, (nome1, nome2) é a forma apropriada de se passar 2 variáveis. Uma série de valores entre parênteses é chamado de \emph{tupla} e é bem parecido com uma lista, que nós abordaremos depois.

\section{Não é igual a uma lista de compras}\index{Listas}

Ovos, leite, queijo, salsinha, margarina e fermento. Que não é uma lista de compras completa, mas serve para o que precisamos. Se você quiser armazenar isso em uma variável, você pode criar uma string:

\begin{listing}
\begin{verbatim}
>>> lista_de_compras = 'ovos, leite, queijo, salsinha, margarina, fermento'
>>> print(lista_de_compras)
ovos, leite, queijo, salsinha, margarina, fermento
\end{verbatim}
\end{listing}

Uma outra forma, seria criar uma `lista', que é um tipo de objeto especial em Python:

\begin{listing}
\begin{verbatim}
>>> lista_de_compras = ['ovos', 'leite', 'queijo', 'salsinha', 'margarina',
... 'fermento' ]
>>> print(lista_de_compras)
['ovos', 'leite', 'queijo', 'salsinha', 'margarina', 'fermento']
\end{verbatim}
\end{listing}

É um pouco mais de digitação, porém mais útil. Nós poderíamos imprimir o terceiro item da lista, usando a sua posição (chamado de índice), usando colchetes []:

\begin{listing}
\begin{verbatim}
>>> print(lista_de_compras[2])
queijo
\end{verbatim}
\end{listing}

Listas iniciam no índice 0 --- então o primeiro item da lista é 0, o segundo é 1 e o terceiro é 2. Isso não faz muito sentido para a maioria das pessoas, mas faz para os programadores. Um pouco a frente, quando você subir mais alguns degraus você vai começar a contar a partir do zero, ao invés do um. Isso deve confundir um pouco o seu irmão ou irmã.
\par
Nós podemos exibir todos os itens, do 3º ao 5º da lista, usando o dois pontos dentro do colchetes:

\begin{listing}
\begin{verbatim}
>>> print(lista_de_compras[2:5])
['queijo', 'salsinha', 'margarina']
\end{verbatim}
\end{listing}

[2:5] é o mesmo que dizer que nós estamos interessados nos itens do índice 2 ao 5 (mas não o incluindo). E por começarmos a contar do 0, o 3º item da lista é o número 2 e o 5º item é o número 4. Listas podem ser usadas para armazenar qualquer tipo de item. Elas podem armazenar números:

\begin{listing}
\begin{verbatim}
>>> minhalista = [1, 2, 5, 10, 20]
\end{verbatim}
\end{listing}

\noindent
E strings:

\begin{listing}
\begin{verbatim}
>>> minhalista = ['a', 'bbb', 'ccccccc', 'ddddddddd']
\end{verbatim}
\end{listing}

\noindent
E misturar números e strings:

\begin{listing}
\begin{verbatim}
>>> minhalista = [1, 2, 'a', 'bbb']
>>> print(minhalista)
[1, 2, 'a', 'bbb']
\end{verbatim}
\end{listing}

\noindent
E até listas de listas:

\begin{listing}
\begin{verbatim}
>>> lista1 = ['a', 'b', 'c']
>>> lista2 = [1, 2, 3]
>>> minhalista = [lista1, lista2]
>>> print(minhalista)
[['a', 'b', 'c'], [1, 2, 3]]
\end{verbatim}
\end{listing}

No exemplo acima, ama variável chamada `lista' é criada contendo 3 letras, `lista2' é criada contendo 3 números e `minhalista' é criada contendo a `lista1' e a `lista2'. As coisas podem ficar um pouco confusas, rapidamente, se você começar a criar listas de listas de listas$\ldots$ mas felizmente não há muita necessidade de se fazer coisas desse tipo em Python. Ainda assim, é útil para você saber que pode armazenar quaisquer tipos de itens em uma lista em Python.

\noindent
\emph{E não somente a sua lista de compras.}

\subsection*{\color{BrickRed}Substituindo itens}\index{Listas!Substituição}

Nós podemos substituir um item em uma lista, apenas trocando o seu valor, da mesma forma que é trocado o valor de uma variável. Por exemplo, nós podemos trocar a salsinha por alface, trocando o valor do índice 3:

\begin{listing}
\begin{verbatim}
>>> lista_de_compras[3] = 'alface'
>>> print(lista_de_compras)
['ovos', 'leite', 'queijo', 'alface', 'margarina', 'fermento']
\end{verbatim}
\end{listing}

\subsection*{\color{BrickRed}Adicionando itens...}\index{Listas!Adição}

Nós podemos adicionar itens à uma lista, usando o método `append'. Um método é uma ação ou comando que diz ao Python que nós queremos fazer algo. Nós falaremos mais adiante sobre métodos, por agora, para adicionar um item à nossa lista de compras, nós fazemos:

\begin{listing}
\begin{verbatim}
>>> lista_de_compras.append('barra de chocolate')
>>> print(lista_de_compras)
['ovos', 'leite', 'queijo', 'alface', 'margarina', 'fermento', 'barra de chocolate']
\end{verbatim}
\end{listing}

Que certamente torna a lista de compras muito melhor.

\subsection*{\color{BrickRed}$\ldots$e removendo itens}\index{Listas!Remoção}

Nós podemos remover itens de uma lista, usando o comando `del' (nome curto para delete). Por exemplo, para remover o quinto item da lista (fermento):

\begin{listing}
\begin{verbatim}
>>> del lista_de_compras[5]
>>> print(lista_de_compras)
['ovos', 'leite', 'queijo', 'alface', 'margarina', 'barra de chocolate']
\end{verbatim}
\end{listing}

Lembre-se que as posições iniciam em zero, então lista\_de\_compras[5] se refere ao sexto item.

\subsection*{\color{BrickRed}Duas listas é melhor que apenas 1}\index{Listas!Junção}

Nós podemos unir listas, apenas somando-as como se fossem dois números:

\begin{listing}
\begin{verbatim}
>>> lista1 = [1, 2, 3]
>>> lista2 = [4, 5, 6]
>>> print(lista1 + lista2)
[1, 2, 3, 4, 5, 6]
\end{verbatim}
\end{listing}

\noindent
Nós podemos também associar o resultado à uma outra variável:

\begin{listing}
\begin{verbatim}
>>> lista1 = [1, 2, 3]
>>> lista2 = [4, 5, 6]
>>> lista3 = lista1 + lista2
>>> print(lista3)
[1, 2, 3, 4, 5, 6]
\end{verbatim}
\end{listing}

\noindent
E você pode multiplicar uma lista, da mesma forma que nós multiplicamos uma string:

\begin{listing}
\begin{verbatim}
>>> lista1 = [1, 2]
>>> print(lista1 * 5)
[1, 2, 1, 2, 1, 2, 1, 2, 1, 2]
\end{verbatim}
\end{listing}

\noindent
No exemplo acima, multiplicando a lista1 por cinco, é outra forma de se dizer ``repita a lista1 cinco vezes''. Porém, a divisão (/) e a subtração (-) não fazem sentido, quando se trabalhando com listas, então você verá erros quando tentar os seguintes exemplos:

\begin{listing}
\begin{verbatim}
>>> lista1 / 20
Traceback (most recent call last):
  File "<stdin>", line 1, in <module>
TypeError: unsupported operand type(s) for /: 'list' and 'int'
\end{verbatim}
\end{listing}

\noindent
ou:

\begin{listing}
\begin{verbatim}
>>> lista1 - 20
Traceback (most recent call last):
  File "<stdin>", line 1, in <module>
TypeError: unsupported operand type(s) for -: 'type' and 'int'
\end{verbatim}
\end{listing}

\noindent
Você receberá uma mensagem de erro bastante desagradável.

\section{Tuplas e Listas}\label{tuplesandlists}\index{Tuplas}

Uma tupla (mencionada anteriormente) se parece muito com uma lista, mas ao invés de usar colchetes, você usa parênteses --- ex. `(' e `)'. Você pode usar tuplas da mesma forma que usa uma lista:

\begin{listing}
\begin{verbatim}
>>> t = (1, 2, 3)
>>> print(t[1])
2
\end{verbatim}
\end{listing}

A principal diferença é que, diferente das listas, as tuplas não mudam após serem criadas. Então se você tentar substituir um valor, assim como nós fizemos com a lista anteriormente, você verá uma mensagem de erro:

\begin{listing}
\begin{verbatim}
>>> t[0] = 4
Traceback (most recent call last):
  File "<stdin>", line 1, in ?
TypeError: 'tuple' object does not support item assignment
\end{verbatim}
\end{listing}

Isso não significa que você não pode alterar o valor da variável contendo a tupla para um outro valor qualquer. Por exemplo, esse código funcionará:

\begin{listing}
\begin{verbatim}
>>> minhavariavel = (1, 2, 3)
>>> minhavariavel = ['uma', 'lista', 'de', 'strings']
\end{verbatim}
\end{listing}

Primeiro nós criamos a variável \code{minhavariavel} apontando para uma tupla de 3 números. Então nós alteramos \code{minhavariavel} para apontar para uma lista de strings. Isso pode ser um pouco confuso inicialmente. Mas pense nisso como os armários de uma escola. Cada armário possui uma etiqueta com um nome. Você coloca algo no armário, fecha a porta, tranca e tira a chave. Você então tira a etiqueta, vai até outro armário vazio e coloca algo dentro (só que agora você deixa a chave). A tupla é como um armário trancado. Você não pode mudar o que está dentro dele, mas você pode retirar a etiqueta e colar em outro armário destrancado, que você pode colocar e retirar coisas de dentro --- como uma lista.

\section{Coisas para tentar}

\emph{Neste capítulo, nós vismos como calcular simples equações matemáticas usando o terminal do Python. Nós também vimos como os parênteses podem alterar o resultado de uma equação, controlando a ordem em que os operadores são usados. Nós vimos como dizer ao Python para se lembrar de alguns valores para usarmos mais tarde -- usando variáveis --- e também como o Python usa strings para armazenar texto e listas e tuplas para armazenar mais de um item.}
\par

\subsection*{Exercício 1}
Faça uma lista com os seus brinquedos favoritos, com o nome \code{brinquedos}. Faça uma lista com as suas comidas favoritas, com o nome \code{comidas}. Junte as duas listas e nomeie o resultado de \code{favoritos}. Por fim, imprima a variável \code{favoritos}.

\subsection*{Exercício 2}
Se você tem 3 caixas contendo 25 chocolates e 10 caixas contendo 32 doces, quantos doces e chocolates você tem no total? (Dica: você pode fazer com apenas uma equação, no terminal do Python)

\subsection*{Exercício 3}
Crie variáveis para o seu primeiro e último nome. Agora crie uma string e use os sinalizadores para incluir o seu nome.


\newpage
