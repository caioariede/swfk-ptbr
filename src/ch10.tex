% ch10.tex
% This work is licensed under the Creative Commons Attribution-Noncommercial-Share Alike 3.0 New Zealand License.
% To view a copy of this license, visit http://creativecommons.org/licenses/by-nc-sa/3.0/nz
% or send a letter to Creative Commons, 171 Second Street, Suite 300, San Francisco, California, 94105, USA.


\chapter{Por onde seguir a partir daqui}

Parabéns! Você chegou até o final.
\par
O que você aprendeu com esse livro, são os conceitos básicos que farão você aprender outras linguagens de programação mais facilmente. Enquanto o Python é uma linguagem de programação brilhante, uma única linguagem \emph{nem sempre} é a melhor ferramente para qualquer tarefa. Então não tenha medo de aprender outras formas de programar, se isso lhe interessa.

Por exemplo, se você se interessa por programação de jogos, você talvez possa olhar algo como o BlitzBasic (\href{http://www.blitzbasic.com}{www.blitzbasic.com}), que usa a linguagem de programação Basic. Ou talvez o Flash (que é usado em muitos sites para animação e jogos --- por exemplo, o site da Nickelodeon, \href{http://www.nick.com}{www.nick.com}, usa muito Flash).

Se voc6e for interessado em programação de jogos em Flash, possivelmente um bom começo seria o livro `Beginning Flash Games Programming for Dummies', um livro escrito por Andy Harris, ou uma referência mais avançada como o `The Flash 8 Game Developing Handbook' por Serge Melnikov. Procurando por `flash games' na \href{http://www.amazon.com}{www.amazon.com}, você encontrará diversos livros neste assunto.

Outros livros de programação de jogos são: `Beginner's Guide to DarkBASIC Game Programming' por Jonathon S Harbour (também usando a linguagem de programação Basic), e `Game Programming for Teens' por Maneesh Sethi (usando BlitzBasic). Saiba que o BlitzBasic, DarkBasic e Flash (pelo menos as ferramentas de desenvolvimento) são todas pagas (diferente do Python), então a sua mãe ou pai terão que agir antes de você poder começar.

Se você quer continuar com o Python para programar jogos, existe diversos lugares para olhar: \href{http://www.pygame.org}{www.pygame.org} e o livro `Game Programming With Python' por Sean Riley.

Se você não está interessado especificamente em programação de jogos, mas quer aprender um pouco mais sobre Python (mais tópicos avançados de programação), então dê uma olhada no livro `Dive into Python' por Mark Pilgrim (\href{http://www.diveintopython.org}{www.diveintopython.org}). Também existe um bom tutorial gratuito para Python disponível em: \href{http://docs.python.org/tut/tut.html}{http://docs.python.org/tut/tut.html}. Existe uma pilha de tópicos que nós não cobrimos nessa introdução básica, então, pelo menos da perspectiva do Python, existe muita coisa para se aprender e brincar.
\par\par\noindent
\emph{Boa sorte e desfrute dos seus aprendizados de programação.}

\newpage
